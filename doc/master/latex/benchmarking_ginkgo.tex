In addition to the unit tests designed to verify correctness, Ginkgo also includes a benchmark suite for checking its performance on the system. To compile the benchmarks, the flag {\ttfamily -\/\+D\+G\+I\+N\+K\+G\+O\+\_\+\+B\+U\+I\+L\+D\+\_\+\+B\+E\+N\+C\+H\+M\+A\+R\+KS=ON} has to be set during the {\ttfamily cmake} step. In addition, the \href{https://github.com/ginkgo-project/ssget}{\tt {\ttfamily ssget} command-\/line utility} has to be installed on the system.

The benchmark suite tests Ginkgo\textquotesingle{}s performance using the \href{https://sparse.tamu.edu/}{\tt Suite\+Sparse matrix collection} and artificially generated matrices. The suite sparse collection will be downloaded automatically when the benchmarks are run. Please note that the entire collection requires roughly 100\+GB of disk storage in its compressed format, and roughly 25\+GB of additional disk space for intermediate data (such us uncompressing the archive). Additionally, the benchmark runs usually take a long time (Sp\+MV benchmarks on the complete collection take roughly 24h using the K20 G\+PU), and will stress the system.

The benchmark suite is invoked using the {\ttfamily make benchmark} command in the build directory. The behavior of the suite can be modified using environment variables. Assuming the {\ttfamily bash} shell is used, these can either be specified via the {\ttfamily export} command to persist between multiple runs\+:


\begin{DoxyCode}
export VARIABLE="value"
...
make benchmark
\end{DoxyCode}


or specified on the fly, on the same line as the {\ttfamily make benchmark} command\+:


\begin{DoxyCode}
env VARIABLE="value" ... make benchmark
\end{DoxyCode}


Since {\ttfamily make} sets any variables passed to it as temporary environment variables, the following shorthand can also be used\+:


\begin{DoxyCode}
make benchmark VARIABLE="value" ...
\end{DoxyCode}


A combination of the above approaches is also possible (e.\+g. it may be useful to {\ttfamily export} the {\ttfamily S\+Y\+S\+T\+E\+M\+\_\+\+N\+A\+ME} variable, and specify the others at every benchmark run).

Supported environment variables are described in the following list\+:


\begin{DoxyItemize}
\item {\ttfamily B\+E\+N\+C\+H\+M\+A\+RK=\{spmv, solver, preconditioner\}} -\/ The benchmark set to run. Default is {\ttfamily spmv}.
\begin{DoxyItemize}
\item {\ttfamily spmv} -\/ Runs the sparse matrix-\/vector product benchmarks on the Suite\+Sparse collection.
\item {\ttfamily solver} -\/ Runs the solver benchmarks on the Suite\+Sparse collection. The matrix format is determined by running the {\ttfamily spmv} benchmarks first, and using the fastest format determined by that benchmark. The maximum number of iterations for the iterative solvers is set to 10,000 and the requested residual reduction factor to 1e-\/6.
\item {\ttfamily preconditioner} -\/ Runs the preconditioner benchmarks on artificially generated block-\/diagonal matrices.
\end{DoxyItemize}
\item {\ttfamily D\+R\+Y\+\_\+\+R\+UN=\{true, false\}} -\/ If set to {\ttfamily true}, prepares the system for the benchmark runs (downloads the collections, creates the result structure, etc.) and outputs the list of commands that would normally be run, but does not run the benchmarks themselves. Default is {\ttfamily false}.
\item {\ttfamily E\+X\+E\+C\+U\+T\+OR=\{reference,cuda,omp\}} -\/ The executor used for running the benchmarks. Default is {\ttfamily cuda}.
\item {\ttfamily S\+E\+G\+M\+E\+N\+TS=$<$N$>$} -\/ Splits the benchmark suite into {\ttfamily $<$N$>$} segments. This option is useful for running the benchmarks on an H\+PC system with a batch scheduler, as it enables partitioning of the benchmark suite and running it concurrently on multiple nodes of the system. If specified, {\ttfamily S\+E\+G\+M\+E\+N\+T\+\_\+\+ID} also has to be set. Default is {\ttfamily 1}.
\item {\ttfamily S\+E\+G\+M\+E\+N\+T\+\_\+\+ID=$<$I$>$} -\/ used in combination with the {\ttfamily S\+E\+G\+M\+E\+N\+TS} variable. {\ttfamily $<$I$>$} should be an integer between 1 and {\ttfamily $<$N$>$}. If specified, only the {\ttfamily $<$I$>$}-\/th segment of the benchmark suite will be run. Default is {\ttfamily 1}.
\item {\ttfamily S\+Y\+S\+T\+E\+M\+\_\+\+N\+A\+ME=$<$name$>$} -\/ the name of the system where the benchmarks are being run. This option only changes the directory where the benchmark results are stored. It can be used to avoid overwriting the benchmarks if multiple systems share the same filesystem, or when copying the results between systems. Default is {\ttfamily unknown}.
\end{DoxyItemize}

Once {\ttfamily make benchmark} completes, the results can be found in {\ttfamily $<$Ginkgo build directory$>$/benchmark/results/$<$S\+Y\+S\+T\+E\+M\+\_\+\+N\+A\+ME$>$/}. The files are written in the J\+S\+ON format, and can be analyzed using any of the data analysis tools that support J\+S\+ON. Alternatively, they can be uploaded to an online repository, and analyzed using Ginkgo\textquotesingle{}s free web tool \href{https://ginkgo-project.github.io/gpe/}{\tt Ginkgo Performance Explorer (G\+PE)}. (Make sure to change the \char`\"{}\+Performance data U\+R\+L\char`\"{} to your repository if using G\+PE.) 