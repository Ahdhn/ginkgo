The ranges and accessor example.

 \label{_Intro}%
 \label{_Introduction}%
\section*{Introduction}

\label{_Abouttheexample}%
\subsubsection*{About the example }

\label{_CommProg}%
 \section*{The commented program}


\begin{DoxyCode}
\textcolor{preprocessor}{#include <ginkgo/ginkgo.hpp>}
\textcolor{preprocessor}{#include <iomanip>}
\textcolor{preprocessor}{#include <iostream>}
\end{DoxyCode}


LU factorization implementation using Ginkgo ranges For simplicity, we only consider square matrices, and no pivoting.


\begin{DoxyCode}
\textcolor{keyword}{template} <\textcolor{keyword}{typename} Accessor>
\textcolor{keywordtype}{void} factorize(\textcolor{keyword}{const} \hyperlink{classgko_1_1range}{gko::range<Accessor>} &A)
\end{DoxyCode}


note\+: const means that the range (i.\+e. the data handler) is constant, not that the underlying data is constant!


\begin{DoxyCode}
\{
    \textcolor{keyword}{using} \hyperlink{structgko_1_1span}{gko::span};
    assert(A.\hyperlink{classgko_1_1range_a3ac8b238c377da9cc05d7c728efabfc8}{length}(0) == A.\hyperlink{classgko_1_1range_a3ac8b238c377da9cc05d7c728efabfc8}{length}(1));
    \textcolor{keywordflow}{for} (\hyperlink{namespacegko_a6e5c95df0ae4e47aab2f604a22d98ee7}{gko::size\_type} i = 0; i < A.\hyperlink{classgko_1_1range_a3ac8b238c377da9cc05d7c728efabfc8}{length}(0) - 1; ++i) \{
        \textcolor{keyword}{const} \textcolor{keyword}{auto} trail = span\{i + 1, A.\hyperlink{classgko_1_1range_a3ac8b238c377da9cc05d7c728efabfc8}{length}(0)\};
\end{DoxyCode}


note\+: neither of the lines below need additional memory to store intermediate arrays, all computation is done at the point of assignment


\begin{DoxyCode}
A(trail, i) = A(trail, i) / A(i, i);
\end{DoxyCode}


caveat\+: operator $\ast$ is element-\/wise multiplication, mmul is matrix multiplication


\begin{DoxyCode}
        A(trail, trail) = A(trail, trail) - mmul(A(trail, i), A(i, trail));
    \}
\}
\end{DoxyCode}


a utility function for printing the factorization on screen


\begin{DoxyCode}
\textcolor{keyword}{template} <\textcolor{keyword}{typename} Accessor>
\textcolor{keywordtype}{void} print\_lu(\textcolor{keyword}{const} \hyperlink{classgko_1_1range}{gko::range<Accessor>} &A)
\{
    std::cout << std::setprecision(2) << std::fixed;
    std::cout << \textcolor{stringliteral}{"L = ["};
    \textcolor{keywordflow}{for} (\textcolor{keywordtype}{int} i = 0; i < A.\hyperlink{classgko_1_1range_a3ac8b238c377da9cc05d7c728efabfc8}{length}(0); ++i) \{
        std::cout << \textcolor{stringliteral}{"\(\backslash\)n  "};
        \textcolor{keywordflow}{for} (\textcolor{keywordtype}{int} j = 0; j < A.\hyperlink{classgko_1_1range_a3ac8b238c377da9cc05d7c728efabfc8}{length}(1); ++j) \{
            std::cout << (i > j ? A(i, j) : (i == j) * 1.) << \textcolor{stringliteral}{" "};
        \}
    \}
    std::cout << \textcolor{stringliteral}{"\(\backslash\)n]\(\backslash\)n\(\backslash\)nU = ["};
    \textcolor{keywordflow}{for} (\textcolor{keywordtype}{int} i = 0; i < A.\hyperlink{classgko_1_1range_a3ac8b238c377da9cc05d7c728efabfc8}{length}(0); ++i) \{
        std::cout << \textcolor{stringliteral}{"\(\backslash\)n  "};
        \textcolor{keywordflow}{for} (\textcolor{keywordtype}{int} j = 0; j < A.\hyperlink{classgko_1_1range_a3ac8b238c377da9cc05d7c728efabfc8}{length}(1); ++j) \{
            std::cout << (i <= j ? A(i, j) : 0.) << \textcolor{stringliteral}{" "};
        \}
    \}
    std::cout << \textcolor{stringliteral}{"\(\backslash\)n]"} << std::endl;
\}


\textcolor{keywordtype}{int} main(\textcolor{keywordtype}{int} argc, \textcolor{keywordtype}{char} *argv[])
\{
\end{DoxyCode}


Print version information


\begin{DoxyCode}
std::cout << \hyperlink{classgko_1_1version__info_a6daeb8a087cfb57fa055526fc133d8eb}{gko::version\_info::get}() << std::endl;
\end{DoxyCode}


Create some test data, add some padding just to demonstrate how to use it with ranges. clang-\/format off


\begin{DoxyCode}
\textcolor{keywordtype}{double} data[] = \{
    2.,  4.,  5., -1.0,
    4., 11., 12., -1.0,
    6., 24., 24., -1.0
\};
\end{DoxyCode}


clang-\/format on

Create a 3-\/by-\/3 range, with a 2D row-\/major accessor using data as the underlying storage. Set the stride (a.\+k.\+a. \char`\"{}\+L\+D\+A\char`\"{}) to 4.


\begin{DoxyCode}
\textcolor{keyword}{auto} A = \hyperlink{classgko_1_1range}{gko::range<gko::accessor::row\_major<double, 2>}>(data
      , 3u, 3u, 4u);
\end{DoxyCode}


use the LU factorization routine defined above to factorize the matrix


\begin{DoxyCode}
factorize(A);
\end{DoxyCode}


print the factorization on screen


\begin{DoxyCode}
    print\_lu(A);
\}
\end{DoxyCode}
 \label{_Results}%
\section*{Results}

This is the expected output\+:


\begin{DoxyCode}
This is Ginkgo 0.0.0 (develop)
    running with core module 0.0.0 (develop)
    the reference module is  0.0.0 (develop)
    the OpenMP    module is  0.0.0 (develop)
    the CUDA      module is  not compiled
L = [
  1.00 0.00 0.00 
  2.00 1.00 0.00 
  3.00 4.00 1.00 
]

U = [
  2.00 4.00 5.00 
  0.00 3.00 2.00 
  0.00 0.00 1.00 
]
\end{DoxyCode}


\label{_Commentsaboutprogramminganddebugging}%
\paragraph*{Comments about programming and debugging }

\label{_PlainProg}%
 \section*{The plain program}


\begin{DoxyCodeInclude}
\textcolor{comment}{/*******************************<GINKGO LICENSE>******************************}
\textcolor{comment}{Copyright (c) 2017-2019, the Ginkgo authors}
\textcolor{comment}{All rights reserved.}
\textcolor{comment}{}
\textcolor{comment}{Redistribution and use in source and binary forms, with or without}
\textcolor{comment}{modification, are permitted provided that the following conditions}
\textcolor{comment}{are met:}
\textcolor{comment}{}
\textcolor{comment}{1. Redistributions of source code must retain the above copyright}
\textcolor{comment}{notice, this list of conditions and the following disclaimer.}
\textcolor{comment}{}
\textcolor{comment}{2. Redistributions in binary form must reproduce the above copyright}
\textcolor{comment}{notice, this list of conditions and the following disclaimer in the}
\textcolor{comment}{documentation and/or other materials provided with the distribution.}
\textcolor{comment}{}
\textcolor{comment}{3. Neither the name of the copyright holder nor the names of its}
\textcolor{comment}{contributors may be used to endorse or promote products derived from}
\textcolor{comment}{this software without specific prior written permission.}
\textcolor{comment}{}
\textcolor{comment}{THIS SOFTWARE IS PROVIDED BY THE COPYRIGHT HOLDERS AND CONTRIBUTORS "AS}
\textcolor{comment}{IS" AND ANY EXPRESS OR IMPLIED WARRANTIES, INCLUDING, BUT NOT LIMITED}
\textcolor{comment}{TO, THE IMPLIED WARRANTIES OF MERCHANTABILITY AND FITNESS FOR A}
\textcolor{comment}{PARTICULAR PURPOSE ARE DISCLAIMED. IN NO EVENT SHALL THE COPYRIGHT}
\textcolor{comment}{HOLDER OR CONTRIBUTORS BE LIABLE FOR ANY DIRECT, INDIRECT, INCIDENTAL,}
\textcolor{comment}{SPECIAL, EXEMPLARY, OR CONSEQUENTIAL DAMAGES (INCLUDING, BUT NOT}
\textcolor{comment}{LIMITED TO, PROCUREMENT OF SUBSTITUTE GOODS OR SERVICES; LOSS OF USE,}
\textcolor{comment}{DATA, OR PROFITS; OR BUSINESS INTERRUPTION) HOWEVER CAUSED AND ON ANY}
\textcolor{comment}{THEORY OF LIABILITY, WHETHER IN CONTRACT, STRICT LIABILITY, OR TORT}
\textcolor{comment}{(INCLUDING NEGLIGENCE OR OTHERWISE) ARISING IN ANY WAY OUT OF THE USE}
\textcolor{comment}{OF THIS SOFTWARE, EVEN IF ADVISED OF THE POSSIBILITY OF SUCH DAMAGE.}
\textcolor{comment}{******************************<GINKGO LICENSE>*******************************/}

\textcolor{preprocessor}{#include <ginkgo/ginkgo.hpp>}
\textcolor{preprocessor}{#include <iomanip>}
\textcolor{preprocessor}{#include <iostream>}


\textcolor{keyword}{template} <\textcolor{keyword}{typename} Accessor>
\textcolor{keywordtype}{void} factorize(\textcolor{keyword}{const} \hyperlink{classgko_1_1range}{gko::range<Accessor>} &A)
\{
    \textcolor{keyword}{using} \hyperlink{structgko_1_1span}{gko::span};
    assert(A.\hyperlink{classgko_1_1range_a3ac8b238c377da9cc05d7c728efabfc8}{length}(0) == A.\hyperlink{classgko_1_1range_a3ac8b238c377da9cc05d7c728efabfc8}{length}(1));
    \textcolor{keywordflow}{for} (\hyperlink{namespacegko_a6e5c95df0ae4e47aab2f604a22d98ee7}{gko::size\_type} i = 0; i < A.\hyperlink{classgko_1_1range_a3ac8b238c377da9cc05d7c728efabfc8}{length}(0) - 1; ++i) \{
        \textcolor{keyword}{const} \textcolor{keyword}{auto} trail = span\{i + 1, A.\hyperlink{classgko_1_1range_a3ac8b238c377da9cc05d7c728efabfc8}{length}(0)\};
        A(trail, i) = A(trail, i) / A(i, i);
        A(trail, trail) = A(trail, trail) - mmul(A(trail, i), A(i, trail));
    \}
\}


\textcolor{keyword}{template} <\textcolor{keyword}{typename} Accessor>
\textcolor{keywordtype}{void} print\_lu(\textcolor{keyword}{const} \hyperlink{classgko_1_1range}{gko::range<Accessor>} &A)
\{
    std::cout << std::setprecision(2) << std::fixed;
    std::cout << \textcolor{stringliteral}{"L = ["};
    \textcolor{keywordflow}{for} (\textcolor{keywordtype}{int} i = 0; i < A.\hyperlink{classgko_1_1range_a3ac8b238c377da9cc05d7c728efabfc8}{length}(0); ++i) \{
        std::cout << \textcolor{stringliteral}{"\(\backslash\)n  "};
        \textcolor{keywordflow}{for} (\textcolor{keywordtype}{int} j = 0; j < A.\hyperlink{classgko_1_1range_a3ac8b238c377da9cc05d7c728efabfc8}{length}(1); ++j) \{
            std::cout << (i > j ? A(i, j) : (i == j) * 1.) << \textcolor{stringliteral}{" "};
        \}
    \}
    std::cout << \textcolor{stringliteral}{"\(\backslash\)n]\(\backslash\)n\(\backslash\)nU = ["};
    \textcolor{keywordflow}{for} (\textcolor{keywordtype}{int} i = 0; i < A.\hyperlink{classgko_1_1range_a3ac8b238c377da9cc05d7c728efabfc8}{length}(0); ++i) \{
        std::cout << \textcolor{stringliteral}{"\(\backslash\)n  "};
        \textcolor{keywordflow}{for} (\textcolor{keywordtype}{int} j = 0; j < A.\hyperlink{classgko_1_1range_a3ac8b238c377da9cc05d7c728efabfc8}{length}(1); ++j) \{
            std::cout << (i <= j ? A(i, j) : 0.) << \textcolor{stringliteral}{" "};
        \}
    \}
    std::cout << \textcolor{stringliteral}{"\(\backslash\)n]"} << std::endl;
\}


\textcolor{keywordtype}{int} main(\textcolor{keywordtype}{int} argc, \textcolor{keywordtype}{char} *argv[])
\{
    std::cout << \hyperlink{classgko_1_1version__info_a6daeb8a087cfb57fa055526fc133d8eb}{gko::version\_info::get}() << std::endl;

    \textcolor{keywordtype}{double} data[] = \{
        2.,  4.,  5., -1.0,
        4., 11., 12., -1.0,
        6., 24., 24., -1.0
    \};

    \textcolor{keyword}{auto} A = \hyperlink{classgko_1_1range}{gko::range<gko::accessor::row\_major<double, 2>}>(
      data, 3u, 3u, 4u);

    factorize(A);

    print\_lu(A);
\}
\end{DoxyCodeInclude}
 