The external library(deal.\+II) interfacing example.

 \label{_Intro}%
 \label{_Introduction}%
\section*{Introduction}

\label{_Abouttheexample}%
\subsubsection*{About the example }

\label{_CommProg}%
 \section*{The commented program}


\begin{DoxyCode}
/ * ---------------------------------------------------------------------
 *
 * Copyright (C) 2000 - 2018 by the deal.II authors
 *
 * This file is part of the deal.II library.
 *
 * The deal.II library is free software; you can use it, redistribute
 * it, and/or modify it under the terms of the GNU Lesser General
 * Public License as published by the Free Software Foundation; either
 * version 2.1 of the License, or (at your option) any later version.
 * The full text of the license can be found in the file LICENSE at
 * the top level of the deal.II distribution.
 *
 * ---------------------------------------------------------------------

 *
 * Author: Wolfgang Bangerth, University of Heidelberg, 2000
 * /

/ * ---------------------------------------------------------------------
 *
 * This file has been taken verbatim from the deal.ii (version 9.0)
 * examples directory and modified.
 *
 * This example aims to demonstrate the ease with which Ginkgo can
 * be interfaced with other libraries. The only modification/ addition
 * has been to the AdvectionProblem::solve () function.
 *
 * /
\end{DoxyCode}


Just as in previous examples, we have to include several files of which the meaning has already been discussed\+:


\begin{DoxyCode}
\textcolor{preprocessor}{#include <deal.II/base/function.h>}
\textcolor{preprocessor}{#include <deal.II/base/logstream.h>}
\textcolor{preprocessor}{#include <deal.II/base/quadrature\_lib.h>}
\textcolor{preprocessor}{#include <deal.II/dofs/dof\_accessor.h>}
\textcolor{preprocessor}{#include <deal.II/dofs/dof\_handler.h>}
\textcolor{preprocessor}{#include <deal.II/dofs/dof\_tools.h>}
\textcolor{preprocessor}{#include <deal.II/fe/fe\_q.h>}
\textcolor{preprocessor}{#include <deal.II/fe/fe\_values.h>}
\textcolor{preprocessor}{#include <deal.II/grid/grid\_generator.h>}
\textcolor{preprocessor}{#include <deal.II/grid/grid\_out.h>}
\textcolor{preprocessor}{#include <deal.II/grid/grid\_refinement.h>}
\textcolor{preprocessor}{#include <deal.II/grid/tria.h>}
\textcolor{preprocessor}{#include <deal.II/grid/tria\_accessor.h>}
\textcolor{preprocessor}{#include <deal.II/grid/tria\_iterator.h>}
\textcolor{preprocessor}{#include <deal.II/lac/constraint\_matrix.h>}
\textcolor{preprocessor}{#include <deal.II/lac/dynamic\_sparsity\_pattern.h>}
\textcolor{preprocessor}{#include <deal.II/lac/full\_matrix.h>}
\textcolor{preprocessor}{#include <deal.II/lac/precondition.h>}
\textcolor{preprocessor}{#include <deal.II/lac/solver\_bicgstab.h>}
\textcolor{preprocessor}{#include <deal.II/lac/sparse\_matrix.h>}
\textcolor{preprocessor}{#include <deal.II/lac/vector.h>}
\textcolor{preprocessor}{#include <deal.II/numerics/data\_out.h>}
\textcolor{preprocessor}{#include <deal.II/numerics/matrix\_tools.h>}
\textcolor{preprocessor}{#include <deal.II/numerics/vector\_tools.h>}
\end{DoxyCode}


The following two files provide classes and information for multithreaded programs. In the first one, the classes and functions are declared which we need to do assembly in parallel (i.\+e. the {\ttfamily Work\+Stream} namespace). The second file has a class Multithread\+Info which can be used to query the number of processors in your system, which is often useful when deciding how many threads to start in parallel.


\begin{DoxyCode}
\textcolor{preprocessor}{#include <deal.II/base/multithread\_info.h>}
\textcolor{preprocessor}{#include <deal.II/base/work\_stream.h>}
\end{DoxyCode}


The next new include file declares a base class {\ttfamily Tensor\+Function} not unlike the {\ttfamily Function} class, but with the difference that the return value is tensor-\/valued rather than scalar of vector-\/valued.


\begin{DoxyCode}
\textcolor{preprocessor}{#include <deal.II/base/tensor\_function.h>}

\textcolor{preprocessor}{#include <deal.II/numerics/error\_estimator.h>}
\end{DoxyCode}


Ginkgo\textquotesingle{}s header file


\begin{DoxyCode}
\textcolor{preprocessor}{#include <ginkgo/ginkgo.hpp>}
\end{DoxyCode}


This is C++, as we want to write some output to disk\+:


\begin{DoxyCode}
\textcolor{preprocessor}{#include <fstream>}
\textcolor{preprocessor}{#include <iostream>}
\end{DoxyCode}


The last step is as in previous programs\+:


\begin{DoxyCode}
\textcolor{keyword}{namespace }Step9 \{
\textcolor{keyword}{using namespace }dealii;
\end{DoxyCode}


\label{_AdvectionProblemclassdeclaration}%
 \subsubsection*{Advection\+Problem class declaration}

Following we declare the main class of this program. It is very much like the main classes of previous examples, so we again only comment on the differences.


\begin{DoxyCode}
\textcolor{keyword}{template} <\textcolor{keywordtype}{int} dim>
\textcolor{keyword}{class }AdvectionProblem \{
\textcolor{keyword}{public}:
    AdvectionProblem();
    ~AdvectionProblem();
    \textcolor{keywordtype}{void} run();

\textcolor{keyword}{private}:
    \textcolor{keywordtype}{void} setup\_system();
\end{DoxyCode}


The next set of functions will be used to assemble the matrix. However, unlike in the previous examples, the {\ttfamily assemble\+\_\+system()} function will not do the work itself, but rather will delegate the actual assembly to helper functions {\ttfamily assemble\+\_\+local\+\_\+system()} and {\ttfamily copy\+\_\+local\+\_\+to\+\_\+global()}. The rationale is that matrix assembly can be parallelized quite well, as the computation of the local contributions on each cell is entirely independent of other cells, and we only have to synchronize when we add the contribution of a cell to the global matrix.

The strategy for parallelization we choose here is one of the possibilities mentioned in detail in the threads module in the documentation. Specifically, we will use the Work\+Stream approach discussed there. Since there is so much documentation in this module, we will not repeat the rationale for the design choices here (for example, if you read through the module mentioned above, you will understand what the purpose of the {\ttfamily Assembly\+Scratch\+Data} and {\ttfamily Assembly\+Copy\+Data} structures is). Rather, we will only discuss the specific implementation.

If you read the page mentioned above, you will find that in order to parallelize assembly, we need two data structures -- one that corresponds to data that we need during local integration (\char`\"{}scratch data\char`\"{}, i.\+e., things we only need as temporary storage), and one that carries information from the local integration to the function that then adds the local contributions to the corresponding elements of the global matrix. The former of these typically contains the F\+E\+Values and F\+E\+Face\+Values objects, whereas the latter has the local matrix, local right hand side, and information about which degrees of freedom live on the cell for which we are assembling a local contribution. With this information, the following should be relatively self-\/explanatory\+:


\begin{DoxyCode}
\textcolor{keyword}{struct }AssemblyScratchData \{
    AssemblyScratchData(\textcolor{keyword}{const} FiniteElement<dim> &fe);
    AssemblyScratchData(\textcolor{keyword}{const} AssemblyScratchData &scratch\_data);

    FEValues<dim> fe\_values;
    FEFaceValues<dim> fe\_face\_values;
\};

\textcolor{keyword}{struct }AssemblyCopyData \{
    FullMatrix<double> cell\_matrix;
    Vector<double> cell\_rhs;
    std::vector<types::global\_dof\_index> local\_dof\_indices;
\};

\textcolor{keywordtype}{void} assemble\_system();
\textcolor{keywordtype}{void} local\_assemble\_system(
    \textcolor{keyword}{const} \textcolor{keyword}{typename} DoFHandler<dim>::active\_cell\_iterator &cell,
    AssemblyScratchData &scratch, AssemblyCopyData &copy\_data);
\textcolor{keywordtype}{void} copy\_local\_to\_global(\textcolor{keyword}{const} AssemblyCopyData &copy\_data);
\end{DoxyCode}


The following functions again are as in previous examples, as are the subsequent variables.


\begin{DoxyCode}
    \textcolor{keywordtype}{void} solve();
    \textcolor{keywordtype}{void} refine\_grid();
    \textcolor{keywordtype}{void} output\_results(\textcolor{keyword}{const} \textcolor{keywordtype}{unsigned} \textcolor{keywordtype}{int} cycle) \textcolor{keyword}{const};

    Triangulation<dim> triangulation;
    DoFHandler<dim> dof\_handler;

    FE\_Q<dim> fe;

    ConstraintMatrix hanging\_node\_constraints;

    SparsityPattern sparsity\_pattern;
    SparseMatrix<double> system\_matrix;

    Vector<double> solution;
    Vector<double> system\_rhs;
\};
\end{DoxyCode}


\label{_Equationdatadeclaration}%
 \subsubsection*{Equation data declaration}

Next we declare a class that describes the advection field. This, of course, is a vector field with as many components as there are space dimensions. One could now use a class derived from the {\ttfamily Function} base class, as we have done for boundary values and coefficients in previous examples, but there is another possibility in the library, namely a base class that describes tensor valued functions. In contrast to the usual {\ttfamily Function} objects, we provide the compiler with knowledge on the size of the objects of the return type. This enables the compiler to generate efficient code, which is not so simple for usual vector-\/valued functions where memory has to be allocated on the heap (thus, the {\ttfamily Function\+::vector\+\_\+value} function has to be given the address of an object into which the result is to be written, in order to avoid copying and memory allocation and deallocation on the heap). In addition to the known size, it is possible not only to return vectors, but also tensors of higher rank; however, this is not very often requested by applications, to be honest...

The interface of the {\ttfamily Tensor\+Function} class is relatively close to that of the {\ttfamily Function} class, so there is probably no need to comment in detail the following declaration\+:


\begin{DoxyCode}
\textcolor{keyword}{template} <\textcolor{keywordtype}{int} dim>
\textcolor{keyword}{class }AdvectionField : \textcolor{keyword}{public} TensorFunction<1, dim> \{
\textcolor{keyword}{public}:
    AdvectionField() : TensorFunction<1, dim>() \{\}

    \textcolor{keyword}{virtual} Tensor<1, dim> value(\textcolor{keyword}{const} Point<dim> &p) \textcolor{keyword}{const};

    \textcolor{keyword}{virtual} \textcolor{keywordtype}{void} value\_list(\textcolor{keyword}{const} std::vector<Point<dim>> &points,
                            std::vector<Tensor<1, dim>> &values) \textcolor{keyword}{const};
\end{DoxyCode}


In previous examples, we have used assertions that throw exceptions in several places. However, we have never seen how such exceptions are declared. This can be done as follows\+:


\begin{DoxyCode}
DeclException2(ExcDimensionMismatch, \textcolor{keywordtype}{unsigned} \textcolor{keywordtype}{int}, \textcolor{keywordtype}{unsigned} \textcolor{keywordtype}{int},
               << \textcolor{stringliteral}{"The vector has size "} << arg1 << \textcolor{stringliteral}{" but should have "}
               << arg2 << \textcolor{stringliteral}{" elements."});
\end{DoxyCode}


The syntax may look a little strange, but is reasonable. The format is basically as follows\+: use the name of one of the macros {\ttfamily Decl\+ExceptionN}, where {\ttfamily N} denotes the number of additional parameters which the exception object shall take. In this case, as we want to throw the exception when the sizes of two vectors differ, we need two arguments, so we use {\ttfamily Decl\+Exception2}. The first parameter then describes the name of the exception, while the following declare the data types of the parameters. The last argument is a sequence of output directives that will be piped into the {\ttfamily std\+::cerr} object, thus the strange format with the leading {\ttfamily $<$$<$} operator and the like. Note that we can access the parameters which are passed to the exception upon construction (i.\+e. within the {\ttfamily Assert} call) by using the names {\ttfamily arg1} through {\ttfamily argN}, where {\ttfamily N} is the number of arguments as defined by the use of the respective macro {\ttfamily Decl\+ExceptionN}.

To learn how the preprocessor expands this macro into actual code, please refer to the documentation of the exception classes in the base library. Suffice it to say that by this macro call, the respective exception class is declared, which also has error output functions already implemented.


\begin{DoxyCode}
\};
\end{DoxyCode}


The following two functions implement the interface described above. The first simply implements the function as described in the introduction, while the second uses the same trick to avoid calling a virtual function as has already been introduced in the previous example program. Note the check for the right sizes of the arguments in the second function, which should always be present in such functions; it is our experience that many if not most programming errors result from incorrectly initialized arrays, incompatible parameters to functions and the like; using assertion as in this case can eliminate many of these problems.


\begin{DoxyCode}
\textcolor{keyword}{template} <\textcolor{keywordtype}{int} dim>
Tensor<1, dim> AdvectionField<dim>::value(\textcolor{keyword}{const} Point<dim> &p)\textcolor{keyword}{ const}
\textcolor{keyword}{}\{
    Point<dim> value;
    value[0] = 2;
    \textcolor{keywordflow}{for} (\textcolor{keywordtype}{unsigned} \textcolor{keywordtype}{int} i = 1; i < dim; ++i)
        value[i] = 1 + 0.8 * std::sin(8 * numbers::PI * p[0]);

    \textcolor{keywordflow}{return} value;
\}


\textcolor{keyword}{template} <\textcolor{keywordtype}{int} dim>
\textcolor{keywordtype}{void} AdvectionField<dim>::value\_list(\textcolor{keyword}{const} std::vector<Point<dim>> &points,
                                     std::vector<Tensor<1, dim>> &values)\textcolor{keyword}{ const}
\textcolor{keyword}{}\{
    Assert(values.size() == points.size(),
           ExcDimensionMismatch(values.size(), points.size()));

    \textcolor{keywordflow}{for} (\textcolor{keywordtype}{unsigned} \textcolor{keywordtype}{int} i = 0; i < points.size(); ++i)
        values[i] = AdvectionField<dim>::value(points[i]);
\}
\end{DoxyCode}


Besides the advection field, we need two functions describing the source terms ({\ttfamily right hand side}) and the boundary values. First for the right hand side, which follows the same pattern as in previous examples. As described in the introduction, the source is a constant function in the vicinity of a source point, which we denote by the constant static variable {\ttfamily center\+\_\+point}. We set the values of this center using the same template tricks as we have shown in the step-\/7 example program. The rest is simple and has been shown previously, including the way to avoid virtual function calls in the {\ttfamily value\+\_\+list} function.


\begin{DoxyCode}
\textcolor{keyword}{template} <\textcolor{keywordtype}{int} dim>
\textcolor{keyword}{class }RightHandSide : \textcolor{keyword}{public} Function<dim> \{
\textcolor{keyword}{public}:
    RightHandSide() : Function<dim>() \{\}

    \textcolor{keyword}{virtual} \textcolor{keywordtype}{double} value(\textcolor{keyword}{const} Point<dim> &p,
                         \textcolor{keyword}{const} \textcolor{keywordtype}{unsigned} \textcolor{keywordtype}{int} component = 0) \textcolor{keyword}{const};

    \textcolor{keyword}{virtual} \textcolor{keywordtype}{void} value\_list(\textcolor{keyword}{const} std::vector<Point<dim>> &points,
                            std::vector<double> &values,
                            \textcolor{keyword}{const} \textcolor{keywordtype}{unsigned} \textcolor{keywordtype}{int} component = 0) \textcolor{keyword}{const};

\textcolor{keyword}{private}:
    \textcolor{keyword}{static} \textcolor{keyword}{const} Point<dim> center\_point;
\};


\textcolor{keyword}{template} <>
\textcolor{keyword}{const} Point<1> RightHandSide<1>::center\_point = Point<1>(-0.75);

\textcolor{keyword}{template} <>
\textcolor{keyword}{const} Point<2> RightHandSide<2>::center\_point = Point<2>(-0.75, -0.75);

\textcolor{keyword}{template} <>
\textcolor{keyword}{const} Point<3> RightHandSide<3>::center\_point = Point<3>(-0.75, -0.75, -0.75);
\end{DoxyCode}


The only new thing here is that we check for the value of the {\ttfamily component} parameter. As this is a scalar function, it is obvious that it only makes sense if the desired component has the index zero, so we assert that this is indeed the case. {\ttfamily Exc\+Index\+Range} is a global predefined exception (probably the one most often used, we therefore made it global instead of local to some class), that takes three parameters\+: the index that is outside the allowed range, the first element of the valid range and the one past the last (i.\+e. again the half-\/open interval so often used in the C++ standard library)\+:


\begin{DoxyCode}
\textcolor{keyword}{template} <\textcolor{keywordtype}{int} dim>
\textcolor{keywordtype}{double} RightHandSide<dim>::value(\textcolor{keyword}{const} Point<dim> &p,
                                 \textcolor{keyword}{const} \textcolor{keywordtype}{unsigned} \textcolor{keywordtype}{int} component)\textcolor{keyword}{ const}
\textcolor{keyword}{}\{
    (void)component;
    Assert(component == 0, ExcIndexRange(component, 0, 1));
    \textcolor{keyword}{const} \textcolor{keywordtype}{double} diameter = 0.1;
    \textcolor{keywordflow}{return} ((p - center\_point).norm\_square() < diameter * diameter
                ? .1 / std::pow(diameter, dim)
                : 0);
\}


\textcolor{keyword}{template} <\textcolor{keywordtype}{int} dim>
\textcolor{keywordtype}{void} RightHandSide<dim>::value\_list(\textcolor{keyword}{const} std::vector<Point<dim>> &points,
                                    std::vector<double> &values,
                                    \textcolor{keyword}{const} \textcolor{keywordtype}{unsigned} \textcolor{keywordtype}{int} component)\textcolor{keyword}{ const}
\textcolor{keyword}{}\{
    Assert(values.size() == points.size(),
           ExcDimensionMismatch(values.size(), points.size()));

    \textcolor{keywordflow}{for} (\textcolor{keywordtype}{unsigned} \textcolor{keywordtype}{int} i = 0; i < points.size(); ++i)
        values[i] = RightHandSide<dim>::value(points[i], component);
\}
\end{DoxyCode}


Finally for the boundary values, which is just another class derived from the {\ttfamily Function} base class\+:


\begin{DoxyCode}
\textcolor{keyword}{template} <\textcolor{keywordtype}{int} dim>
\textcolor{keyword}{class }BoundaryValues : \textcolor{keyword}{public} Function<dim> \{
\textcolor{keyword}{public}:
    BoundaryValues() : Function<dim>() \{\}

    \textcolor{keyword}{virtual} \textcolor{keywordtype}{double} value(\textcolor{keyword}{const} Point<dim> &p,
                         \textcolor{keyword}{const} \textcolor{keywordtype}{unsigned} \textcolor{keywordtype}{int} component = 0) \textcolor{keyword}{const};

    \textcolor{keyword}{virtual} \textcolor{keywordtype}{void} value\_list(\textcolor{keyword}{const} std::vector<Point<dim>> &points,
                            std::vector<double> &values,
                            \textcolor{keyword}{const} \textcolor{keywordtype}{unsigned} \textcolor{keywordtype}{int} component = 0) \textcolor{keyword}{const};
\};


\textcolor{keyword}{template} <\textcolor{keywordtype}{int} dim>
\textcolor{keywordtype}{double} BoundaryValues<dim>::value(\textcolor{keyword}{const} Point<dim> &p,
                                  \textcolor{keyword}{const} \textcolor{keywordtype}{unsigned} \textcolor{keywordtype}{int} component)\textcolor{keyword}{ const}
\textcolor{keyword}{}\{
    (void)component;
    Assert(component == 0, ExcIndexRange(component, 0, 1));

    \textcolor{keyword}{const} \textcolor{keywordtype}{double} sine\_term =
        std::sin(16 * numbers::PI * std::sqrt(p.norm\_square()));
    \textcolor{keyword}{const} \textcolor{keywordtype}{double} weight = std::exp(-5 * p.norm\_square()) / std::exp(-5.);
    \textcolor{keywordflow}{return} sine\_term * weight;
\}


\textcolor{keyword}{template} <\textcolor{keywordtype}{int} dim>
\textcolor{keywordtype}{void} BoundaryValues<dim>::value\_list(\textcolor{keyword}{const} std::vector<Point<dim>> &points,
                                     std::vector<double> &values,
                                     \textcolor{keyword}{const} \textcolor{keywordtype}{unsigned} \textcolor{keywordtype}{int} component)\textcolor{keyword}{ const}
\textcolor{keyword}{}\{
    Assert(values.size() == points.size(),
           ExcDimensionMismatch(values.size(), points.size()));

    \textcolor{keywordflow}{for} (\textcolor{keywordtype}{unsigned} \textcolor{keywordtype}{int} i = 0; i < points.size(); ++i)
        values[i] = BoundaryValues<dim>::value(points[i], component);
\}
\end{DoxyCode}


\label{_GradientEstimationclassdeclaration}%
 \subsubsection*{Gradient\+Estimation class declaration}

Now, finally, here comes the class that will compute the difference approximation of the gradient on each cell and weighs that with a power of the mesh size, as described in the introduction. This class is a simple version of the {\ttfamily Derivative\+Approximation} class in the library, that uses similar techniques to obtain finite difference approximations of the gradient of a finite element field, or of higher derivatives.

The class has one public static function {\ttfamily estimate} that is called to compute a vector of error indicators, and a few private functions that do the actual work on all active cells. As in other parts of the library, we follow an informal convention to use vectors of floats for error indicators rather than the common vectors of doubles, as the additional accuracy is not necessary for estimated values.

In addition to these two functions, the class declares two exceptions which are raised when a cell has no neighbors in each of the space directions (in which case the matrix described in the introduction would be singular and can\textquotesingle{}t be inverted), while the other one is used in the more common case of invalid parameters to a function, namely a vector of wrong size.

Two other comments\+: first, the class has no non-\/static member functions or variables, so this is not really a class, but rather serves the purpose of a {\ttfamily namespace} in C++. The reason that we chose a class over a namespace is that this way we can declare functions that are private. This can be done with namespaces as well, if one declares some functions in header files in the namespace and implements these and other functions in the implementation file. The functions not declared in the header file are still in the namespace but are not callable from outside. However, as we have only one file here, it is not possible to hide functions in the present case.

The second comment is that the dimension template parameter is attached to the function rather than to the class itself. This way, you don\textquotesingle{}t have to specify the template parameter yourself as in most other cases, but the compiler can figure its value out itself from the dimension of the DoF handler object that one passes as first argument.

Before jumping into the fray with the implementation, let us also comment on the parallelization strategy. We have already introduced the necessary framework for using the Work\+Stream concept in the declaration of the main class of this program above. We will use it again here. In the current context, this means that we have to define (i) classes for scratch and copy objects, (ii) a function that does the local computation on one cell, and (iii) a function that copies the local result into a global object. Given this general framework, we will, however, deviate from it a bit. In particular, Work\+Stream was generally invented for cases where each local computation on a cell {\itshape adds} to a global object -- for example, when assembling linear systems where we add local contributions into a global matrix and right hand side. Work\+Stream is designed to handle the potential conflict of multiple threads trying to do this addition at the same time, and consequently has to provide for some way to ensure that only thread gets to do this at a time. Here, however, the situation is slightly different\+: we compute contributions from every cell individually, but then all we need to do is put them into an element of an output vector that is unique to each cell. Consequently, there is no risk that the write operations from two cells might conflict, and the elaborate machinery of Work\+Stream to avoid conflicting writes is not necessary. Consequently, what we will do is this\+: We still need a scratch object that holds, for example, the F\+E\+Values object. However, we only create a fake, empty copy data structure. Likewise, we do need the function that computes local contributions, but since it can already put the result into its final location, we do not need a copy-\/local-\/to-\/global function and will instead give the Work\+Stream\+::run() function an empty function object -- the equivalent to a N\+U\+LL function pointer.


\begin{DoxyCode}
\textcolor{keyword}{class }GradientEstimation \{
\textcolor{keyword}{public}:
    \textcolor{keyword}{template} <\textcolor{keywordtype}{int} dim>
    \textcolor{keyword}{static} \textcolor{keywordtype}{void} estimate(\textcolor{keyword}{const} DoFHandler<dim> &dof,
                         \textcolor{keyword}{const} Vector<double> &solution,
                         Vector<float> &error\_per\_cell);

    DeclException2(ExcInvalidVectorLength, \textcolor{keywordtype}{int}, \textcolor{keywordtype}{int},
                   << \textcolor{stringliteral}{"Vector has length "} << arg1 << \textcolor{stringliteral}{", but should have "}
                   << arg2);
    DeclException0(ExcInsufficientDirections);

\textcolor{keyword}{private}:
    \textcolor{keyword}{template} <\textcolor{keywordtype}{int} dim>
    \textcolor{keyword}{struct }EstimateScratchData \{
        EstimateScratchData(\textcolor{keyword}{const} FiniteElement<dim> &fe,
                            \textcolor{keyword}{const} Vector<double> &solution,
                            Vector<float> &error\_per\_cell);
        EstimateScratchData(\textcolor{keyword}{const} EstimateScratchData &data);

        FEValues<dim> fe\_midpoint\_value;
        \textcolor{keyword}{const} Vector<double> &solution;
        Vector<float> &error\_per\_cell;
    \};

    \textcolor{keyword}{struct }EstimateCopyData \{\};

    \textcolor{keyword}{template} <\textcolor{keywordtype}{int} dim>
    \textcolor{keyword}{static} \textcolor{keywordtype}{void} estimate\_cell(
        \textcolor{keyword}{const} \textcolor{keyword}{typename} DoFHandler<dim>::active\_cell\_iterator &cell,
        EstimateScratchData<dim> &scratch\_data,
        \textcolor{keyword}{const} EstimateCopyData &copy\_data);
\};
\end{DoxyCode}


\label{_AdvectionProblemclassimplementation}%
 \subsubsection*{Advection\+Problem class implementation}

Now for the implementation of the main class. Constructor, destructor and the function {\ttfamily setup\+\_\+system} follow the same pattern that was used previously, so we need not comment on these three function\+:


\begin{DoxyCode}
\textcolor{keyword}{template} <\textcolor{keywordtype}{int} dim>
AdvectionProblem<dim>::AdvectionProblem() : dof\_handler(triangulation), fe(1)
\{\}


\textcolor{keyword}{template} <\textcolor{keywordtype}{int} dim>
AdvectionProblem<dim>::~AdvectionProblem()
\{
    dof\_handler.clear();
\}


\textcolor{keyword}{template} <\textcolor{keywordtype}{int} dim>
\textcolor{keywordtype}{void} AdvectionProblem<dim>::setup\_system()
\{
    dof\_handler.distribute\_dofs(fe);
    hanging\_node\_constraints.clear();
    DoFTools::make\_hanging\_node\_constraints(dof\_handler,
                                            hanging\_node\_constraints);
    hanging\_node\_constraints.close();

    DynamicSparsityPattern dsp(dof\_handler.n\_dofs(), dof\_handler.n\_dofs());
    DoFTools::make\_sparsity\_pattern(dof\_handler, dsp, hanging\_node\_constraints,
                                    / *keep\_constrained\_dofs = * / \textcolor{keyword}{true});
    sparsity\_pattern.copy\_from(dsp);

    system\_matrix.reinit(sparsity\_pattern);

    solution.reinit(dof\_handler.n\_dofs());
    system\_rhs.reinit(dof\_handler.n\_dofs());
\}
\end{DoxyCode}


In the following function, the matrix and right hand side are assembled. As stated in the documentation of the main class above, it does not do this itself, but rather delegates to the function following next, utilizing the Work\+Stream concept discussed in threads .

If you have looked through the threads module, you will have seen that assembling in parallel does not take an incredible amount of extra code as long as you diligently describe what the scratch and copy data objects are, and if you define suitable functions for the local assembly and the copy operation from local contributions to global objects. This done, the following will do all the heavy lifting to get these operations done on multiple threads on as many cores as you have in your system\+:


\begin{DoxyCode}
\textcolor{keyword}{template} <\textcolor{keywordtype}{int} dim>
\textcolor{keywordtype}{void} AdvectionProblem<dim>::assemble\_system()
\{
    WorkStream::run(dof\_handler.begin\_active(), dof\_handler.end(), *\textcolor{keyword}{this},
                    &AdvectionProblem::local\_assemble\_system,
                    &AdvectionProblem::copy\_local\_to\_global,
                    AssemblyScratchData(fe), AssemblyCopyData());
\end{DoxyCode}


After the matrix has been assembled in parallel, we still have to eliminate hanging node constraints. This is something that can\textquotesingle{}t be done on each of the threads separately, so we have to do it now. Note also, that unlike in previous examples, there are no boundary conditions to be applied to the system of equations. This, of course, is due to the fact that we have included them into the weak formulation of the problem.


\begin{DoxyCode}
    hanging\_node\_constraints.condense(system\_matrix);
    hanging\_node\_constraints.condense(system\_rhs);
\}
\end{DoxyCode}


As already mentioned above, we need to have scratch objects for the parallel computation of local contributions. These objects contain F\+E\+Values and F\+E\+Face\+Values objects, and so we will need to have constructors and copy constructors that allow us to create them. In initializing them, note first that we use bilinear elements, so\+Gauss formulae with two points in each space direction are sufficient. For the cell terms we need the values and gradients of the shape functions, the quadrature points in order to determine the source density and the advection field at a given point, and the weights of the quadrature points times the determinant of the Jacobian at these points. In contrast, for the boundary integrals, we don\textquotesingle{}t need the gradients, but rather the normal vectors to the cells. This determines which update flags we will have to pass to the constructors of the members of the class\+:


\begin{DoxyCode}
\textcolor{keyword}{template} <\textcolor{keywordtype}{int} dim>
AdvectionProblem<dim>::AssemblyScratchData::AssemblyScratchData(
    \textcolor{keyword}{const} FiniteElement<dim> &fe)
    : fe\_values(fe, QGauss<dim>(2),
                update\_values | update\_gradients | update\_quadrature\_points |
                    update\_JxW\_values),
      fe\_face\_values(fe, QGauss<dim - 1>(2),
                     update\_values | update\_quadrature\_points |
                         update\_JxW\_values | update\_normal\_vectors)
\{\}


\textcolor{keyword}{template} <\textcolor{keywordtype}{int} dim>
AdvectionProblem<dim>::AssemblyScratchData::AssemblyScratchData(
    \textcolor{keyword}{const} AssemblyScratchData &scratch\_data)
    : fe\_values(scratch\_data.fe\_values.get\_fe(),
                scratch\_data.fe\_values.get\_quadrature(),
                update\_values | update\_gradients | update\_quadrature\_points |
                    update\_JxW\_values),
      fe\_face\_values(scratch\_data.fe\_face\_values.get\_fe(),
                     scratch\_data.fe\_face\_values.get\_quadrature(),
                     update\_values | update\_quadrature\_points |
                         update\_JxW\_values | update\_normal\_vectors)
\{\}
\end{DoxyCode}


Now, this is the function that does the actual work. It is not very different from the {\ttfamily assemble\+\_\+system} functions of previous example programs, so we will again only comment on the differences. The mathematical stuff follows closely what we have said in the introduction.

There are a number of points worth mentioning here, though. The first one is that we have moved the F\+E\+Values and F\+E\+Face\+Values objects into the Scratch\+Data object. We have done so because the alternative would have been to simply create one every time we get into this function -- i.\+e., on every cell. It now turns out that the F\+E\+Values classes were written with the explicit goal of moving everything that remains the same from cell to cell into the construction of the object, and only do as little work as possible in F\+E\+Values\+::reinit() whenever we move to a new cell. What this means is that it would be very expensive to create a new object of this kind in this function as we would have to do it for every cell -- exactly the thing we wanted to avoid with the F\+E\+Values class. Instead, what we do is create it only once (or a small number of times) in the scratch objects and then re-\/use it as often as we can.

This begs the question of whether there are other objects we create in this function whose creation is expensive compared to its use. Indeed, at the top of the function, we declare all sorts of objects. The {\ttfamily Advection\+Field}, {\ttfamily Right\+Hand\+Side} and {\ttfamily Boundary\+Values} do not cost much to create, so there is no harm here. However, allocating memory in creating the {\ttfamily rhs\+\_\+values} and similar variables below typically costs a significant amount of time, compared to just accessing the (temporary) values we store in them. Consequently, these would be candidates for moving into the {\ttfamily Assembly\+Scratch\+Data} class. We will leave this as an exercise.


\begin{DoxyCode}
\textcolor{keyword}{template} <\textcolor{keywordtype}{int} dim>
\textcolor{keywordtype}{void} AdvectionProblem<dim>::local\_assemble\_system(
    \textcolor{keyword}{const} \textcolor{keyword}{typename} DoFHandler<dim>::active\_cell\_iterator &cell,
    AssemblyScratchData &scratch\_data, AssemblyCopyData &copy\_data)
\{
\end{DoxyCode}


First of all, we will need some objects that describe boundary values, right hand side function and the advection field. As we will only perform actions on these objects that do not change them, we declare them as constant, which can enable the compiler in some cases to perform additional optimizations.


\begin{DoxyCode}
\textcolor{keyword}{const} AdvectionField<dim> advection\_field;
\textcolor{keyword}{const} RightHandSide<dim> right\_hand\_side;
\textcolor{keyword}{const} BoundaryValues<dim> boundary\_values;
\end{DoxyCode}


Then we define some abbreviations to avoid unnecessarily long lines\+:


\begin{DoxyCode}
\textcolor{keyword}{const} \textcolor{keywordtype}{unsigned} \textcolor{keywordtype}{int} dofs\_per\_cell = fe.dofs\_per\_cell;
\textcolor{keyword}{const} \textcolor{keywordtype}{unsigned} \textcolor{keywordtype}{int} n\_q\_points =
    scratch\_data.fe\_values.get\_quadrature().size();
\textcolor{keyword}{const} \textcolor{keywordtype}{unsigned} \textcolor{keywordtype}{int} n\_face\_q\_points =
    scratch\_data.fe\_face\_values.get\_quadrature().size();
\end{DoxyCode}


We declare cell matrix and cell right hand side...


\begin{DoxyCode}
copy\_data.cell\_matrix.reinit(dofs\_per\_cell, dofs\_per\_cell);
copy\_data.cell\_rhs.reinit(dofs\_per\_cell);
\end{DoxyCode}


... an array to hold the global indices of the degrees of freedom of the cell on which we are presently working...


\begin{DoxyCode}
copy\_data.local\_dof\_indices.resize(dofs\_per\_cell);
\end{DoxyCode}


... and array in which the values of right hand side, advection direction, and boundary values will be stored, for cell and face integrals respectively\+:


\begin{DoxyCode}
std::vector<double> rhs\_values(n\_q\_points);
std::vector<Tensor<1, dim>> advection\_directions(n\_q\_points);
std::vector<double> face\_boundary\_values(n\_face\_q\_points);
std::vector<Tensor<1, dim>> face\_advection\_directions(n\_face\_q\_points);
\end{DoxyCode}


... then initialize the {\ttfamily F\+E\+Values} object...


\begin{DoxyCode}
scratch\_data.fe\_values.reinit(cell);
\end{DoxyCode}


... obtain the values of right hand side and advection directions at the quadrature points...


\begin{DoxyCode}
advection\_field.value\_list(scratch\_data.fe\_values.get\_quadrature\_points(),
                           advection\_directions);
right\_hand\_side.value\_list(scratch\_data.fe\_values.get\_quadrature\_points(),
                           rhs\_values);
\end{DoxyCode}


... set the value of the streamline diffusion parameter as described in the introduction...


\begin{DoxyCode}
\textcolor{keyword}{const} \textcolor{keywordtype}{double} delta = 0.1 * cell->diameter();
\end{DoxyCode}


... and assemble the local contributions to the system matrix and right hand side as also discussed above\+:


\begin{DoxyCode}
\textcolor{keywordflow}{for} (\textcolor{keywordtype}{unsigned} \textcolor{keywordtype}{int} q\_point = 0; q\_point < n\_q\_points; ++q\_point)
    \textcolor{keywordflow}{for} (\textcolor{keywordtype}{unsigned} \textcolor{keywordtype}{int} i = 0; i < dofs\_per\_cell; ++i) \{
        \textcolor{keywordflow}{for} (\textcolor{keywordtype}{unsigned} \textcolor{keywordtype}{int} j = 0; j < dofs\_per\_cell; ++j)
            copy\_data.cell\_matrix(i, j) +=
                ((advection\_directions[q\_point] *
                  scratch\_data.fe\_values.shape\_grad(j, q\_point) *
                  (scratch\_data.fe\_values.shape\_value(i, q\_point) +
                   delta *
                       (advection\_directions[q\_point] *
                        scratch\_data.fe\_values.shape\_grad(i, q\_point)))) *
                 scratch\_data.fe\_values.JxW(q\_point));

        copy\_data.cell\_rhs(i) +=
            ((scratch\_data.fe\_values.shape\_value(i, q\_point) +
              delta * (advection\_directions[q\_point] *
                       scratch\_data.fe\_values.shape\_grad(i, q\_point))) *
             rhs\_values[q\_point] * scratch\_data.fe\_values.JxW(q\_point));
    \}
\end{DoxyCode}


Besides the cell terms which we have built up now, the bilinear form of the present problem also contains terms on the boundary of the domain. Therefore, we have to check whether any of the faces of this cell are on the boundary of the domain, and if so assemble the contributions of this face as well. Of course, the bilinear form only contains contributions from the {\ttfamily inflow} part of the boundary, but to find out whether a certain part of a face of the present cell is part of the inflow boundary, we have to have information on the exact location of the quadrature points and on the direction of flow at this point; we obtain this information using the F\+E\+Face\+Values object and only decide within the main loop whether a quadrature point is on the inflow boundary.


\begin{DoxyCode}
\textcolor{keywordflow}{for} (\textcolor{keywordtype}{unsigned} \textcolor{keywordtype}{int} face = 0; face < GeometryInfo<dim>::faces\_per\_cell;
     ++face)
    \textcolor{keywordflow}{if} (cell->face(face)->at\_boundary()) \{
\end{DoxyCode}


Ok, this face of the present cell is on the boundary of the domain. Just as for the usual F\+E\+Values object which we have used in previous examples and also above, we have to reinitialize the F\+E\+Face\+Values object for the present face\+:


\begin{DoxyCode}
scratch\_data.fe\_face\_values.reinit(cell, face);
\end{DoxyCode}


For the quadrature points at hand, we ask for the values of the inflow function and for the direction of flow\+:


\begin{DoxyCode}
boundary\_values.value\_list(
    scratch\_data.fe\_face\_values.get\_quadrature\_points(),
    face\_boundary\_values);
advection\_field.value\_list(
    scratch\_data.fe\_face\_values.get\_quadrature\_points(),
    face\_advection\_directions);
\end{DoxyCode}


Now loop over all quadrature points and see whether it is on the inflow or outflow part of the boundary. This is determined by a test whether the advection direction points inwards or outwards of the domain (note that the normal vector points outwards of the cell, and since the cell is at the boundary, the normal vector points outward of the domain, so if the advection direction points into the domain, its scalar product with the normal vector must be negative)\+:


\begin{DoxyCode}
\textcolor{keywordflow}{for} (\textcolor{keywordtype}{unsigned} \textcolor{keywordtype}{int} q\_point = 0; q\_point < n\_face\_q\_points; ++q\_point)
    \textcolor{keywordflow}{if} (scratch\_data.fe\_face\_values.normal\_vector(q\_point) *
            face\_advection\_directions[q\_point] <
        0)
\end{DoxyCode}


If the is part of the inflow boundary, then compute the contributions of this face to the global matrix and right hand side, using the values obtained from the F\+E\+Face\+Values object and the formulae discussed in the introduction\+:


\begin{DoxyCode}
            \textcolor{keywordflow}{for} (\textcolor{keywordtype}{unsigned} \textcolor{keywordtype}{int} i = 0; i < dofs\_per\_cell; ++i) \{
                \textcolor{keywordflow}{for} (\textcolor{keywordtype}{unsigned} \textcolor{keywordtype}{int} j = 0; j < dofs\_per\_cell; ++j)
                    copy\_data.cell\_matrix(i, j) -=
                        (face\_advection\_directions[q\_point] *
                         scratch\_data.fe\_face\_values.normal\_vector(
                             q\_point) *
                         scratch\_data.fe\_face\_values.shape\_value(
                             i, q\_point) *
                         scratch\_data.fe\_face\_values.shape\_value(
                             j, q\_point) *
                         scratch\_data.fe\_face\_values.JxW(q\_point));

                copy\_data.cell\_rhs(i) -=
                    (face\_advection\_directions[q\_point] *
                     scratch\_data.fe\_face\_values.normal\_vector(
                         q\_point) *
                     face\_boundary\_values[q\_point] *
                     scratch\_data.fe\_face\_values.shape\_value(i,
                                                             q\_point) *
                     scratch\_data.fe\_face\_values.JxW(q\_point));
            \}
\}
\end{DoxyCode}


Now go on by transferring the local contributions to the system of equations into the global objects. The first step was to obtain the global indices of the degrees of freedom on this cell.


\begin{DoxyCode}
    cell->get\_dof\_indices(copy\_data.local\_dof\_indices);
\}
\end{DoxyCode}


The second function we needed to write was the one that copies the local contributions the previous function has computed and put into the copy data object, into the global matrix and right hand side vector objects. This is essentially what we always had as the last block of code when assembling something on every cell. The following should therefore be pretty obvious\+:


\begin{DoxyCode}
\textcolor{keyword}{template} <\textcolor{keywordtype}{int} dim>
\textcolor{keywordtype}{void} AdvectionProblem<dim>::copy\_local\_to\_global(
    \textcolor{keyword}{const} AssemblyCopyData &copy\_data)
\{
    \textcolor{keywordflow}{for} (\textcolor{keywordtype}{unsigned} \textcolor{keywordtype}{int} i = 0; i < copy\_data.local\_dof\_indices.size(); ++i) \{
        \textcolor{keywordflow}{for} (\textcolor{keywordtype}{unsigned} \textcolor{keywordtype}{int} j = 0; j < copy\_data.local\_dof\_indices.size(); ++j)
            system\_matrix.add(copy\_data.local\_dof\_indices[i],
                              copy\_data.local\_dof\_indices[j],
                              copy\_data.cell\_matrix(i, j));

        system\_rhs(copy\_data.local\_dof\_indices[i]) += copy\_data.cell\_rhs(i);
    \}
\}
\end{DoxyCode}


Following is the function that solves the linear system of equations. As the system is no more symmetric positive definite as in all the previous examples, we can\textquotesingle{}t use the Conjugate Gradients method anymore. Rather, we use a solver that is tailored to nonsymmetric systems like the one at hand, the Bi\+C\+G\+Stab method. As preconditioner, we use the Block Jacobi method.


\begin{DoxyCode}
\textcolor{keyword}{template} <\textcolor{keywordtype}{int} dim>
\textcolor{keywordtype}{void} AdvectionProblem<dim>::solve()
\{
\end{DoxyCode}


Assert that the system be symmetric.


\begin{DoxyCode}
Assert(system\_matrix.m() == system\_matrix.n(), ExcNotQuadratic());
\textcolor{keyword}{auto} num\_rows = system\_matrix.m();
\end{DoxyCode}


Make a copy of the rhs to use with Ginkgo.


\begin{DoxyCode}
std::vector<double> rhs(num\_rows);
std::copy(system\_rhs.begin(), system\_rhs.begin() + num\_rows, rhs.begin());
\end{DoxyCode}


Ginkgo setup Some shortcuts\+: A vector is a Dense matrix with co-\/dimension 1. The matrix is setup in C\+SR. But various formats can be used. Look at Ginkgo\textquotesingle{}s documentation.


\begin{DoxyCode}
\textcolor{keyword}{using} vec = \hyperlink{classgko_1_1matrix_1_1Dense}{gko::matrix::Dense<>};
\textcolor{keyword}{using} mtx = \hyperlink{classgko_1_1matrix_1_1Csr}{gko::matrix::Csr<>};
\textcolor{keyword}{using} bicgstab = \hyperlink{classgko_1_1solver_1_1Bicgstab}{gko::solver::Bicgstab<>};
\textcolor{keyword}{using} bj = \hyperlink{classgko_1_1preconditioner_1_1Jacobi}{gko::preconditioner::Jacobi<>};
\textcolor{keyword}{using} val\_array = \hyperlink{classgko_1_1Array}{gko::Array<double>};
\end{DoxyCode}


Where the code is to be executed. Can be changed to {\ttfamily omp} or {\ttfamily cuda} to run on multiple threads or on gpu\textquotesingle{}s


\begin{DoxyCode}
std::shared\_ptr<gko::Executor> exec = gko::ReferenceExecutor::create();
\end{DoxyCode}


Setup Ginkgo\textquotesingle{}s data structures


\begin{DoxyCode}
\textcolor{keyword}{auto} b = vec::create(exec, \hyperlink{structgko_1_1dim}{gko::dim<2>}(num\_rows, 1),
                     val\_array::view(exec, num\_rows, rhs.data()), 1);
\textcolor{keyword}{auto} x = vec::create(exec, \hyperlink{structgko_1_1dim}{gko::dim<2>}(num\_rows, 1));
\textcolor{keyword}{auto} A = mtx::create(exec, \hyperlink{structgko_1_1dim}{gko::dim<2>}(num\_rows),
                     system\_matrix.n\_nonzero\_elements());
mtx::value\_type *values = A->get\_values();
mtx::index\_type *row\_ptr = A->get\_row\_ptrs();
mtx::index\_type *col\_idx = A->get\_col\_idxs();
\end{DoxyCode}


Convert to standard C\+SR format As deal.\+ii does not expose its system matrix pointers, we construct them individually.


\begin{DoxyCode}
row\_ptr[0] = 0;
\textcolor{keywordflow}{for} (\textcolor{keyword}{auto} row = 1; row <= num\_rows; ++row) \{
    row\_ptr[row] = row\_ptr[row - 1] + system\_matrix.get\_row\_length(row - 1);
\}

std::vector<mtx::index\_type> ptrs(num\_rows + 1);
std::copy(A->get\_row\_ptrs(), A->get\_row\_ptrs() + num\_rows + 1,
          ptrs.begin());
\textcolor{keywordflow}{for} (\textcolor{keyword}{auto} row = 0; row < system\_matrix.m(); ++row) \{
    \textcolor{keywordflow}{for} (\textcolor{keyword}{auto} p = system\_matrix.begin(row); p != system\_matrix.end(row);
         ++p) \{
\end{DoxyCode}


write entry into the first free one for this row


\begin{DoxyCode}
col\_idx[ptrs[row]] = p->column();
values[ptrs[row]] = p->value();
\end{DoxyCode}


then move pointer ahead


\begin{DoxyCode}
        ++ptrs[row];
    \}
\}
\end{DoxyCode}


Ginkgo solve The stopping criteria is set at maximum iterations of 1000 and a reduction factor of 1e-\/12. For other options, refer to Ginkgo\textquotesingle{}s documentation.


\begin{DoxyCode}
\textcolor{keyword}{auto} solver\_gen =
    bicgstab::build()
        .with\_criteria(
            gko::stop::Iteration::build().with\_max\_iters(1000).on(exec),
            \hyperlink{classgko_1_1stop_1_1ResidualNormReduction}{gko::stop::ResidualNormReduction<>::build}()
                .with\_reduction\_factor(1e-12)
                .on(exec))
        .with\_preconditioner(bj::build().on(exec))
        .on(exec);
\textcolor{keyword}{auto} solver = solver\_gen->generate(gko::give(A));
\end{DoxyCode}


Solve system


\begin{DoxyCode}
solver->apply(gko::lend(b), gko::lend(x));
\end{DoxyCode}


Copy the solution vector back to deal.\+ii\textquotesingle{}s data structures.


\begin{DoxyCode}
std::copy(x->get\_values(), x->get\_values() + num\_rows, solution.begin());

/ ******************************************************
 * deal.ii \textcolor{keyword}{internal} solver. Here \textcolor{keywordflow}{for} reference.
 SolverControl           solver\_control (1000, 1e-12);
 SolverBicgstab<>        bicgstab (solver\_control);

 PreconditionJacobi<> preconditioner;
 preconditioner.initialize(system\_matrix, 1.0);

 bicgstab.solve (system\_matrix, solution, system\_rhs,
                 preconditioner);
******************************************************* /
\end{DoxyCode}


Give the solution back to deall.\+ii


\begin{DoxyCode}
    hanging\_node\_constraints.distribute(solution);
\}
\end{DoxyCode}


The following function refines the grid according to the quantity described in the introduction. The respective computations are made in the class {\ttfamily Gradient\+Estimation}. The only difference to previous examples is that we refine a little more aggressively (0.\+5 instead of 0.\+3 of the number of cells).


\begin{DoxyCode}
\textcolor{keyword}{template} <\textcolor{keywordtype}{int} dim>
\textcolor{keywordtype}{void} AdvectionProblem<dim>::refine\_grid()
\{
    Vector<float> estimated\_error\_per\_cell(triangulation.n\_active\_cells());

    GradientEstimation::estimate(dof\_handler, solution,
                                 estimated\_error\_per\_cell);

    GridRefinement::refine\_and\_coarsen\_fixed\_number(
        triangulation, estimated\_error\_per\_cell, 0.5, 0.03);

    triangulation.execute\_coarsening\_and\_refinement();
\}
\end{DoxyCode}


Writing output to disk is done in the same way as in the previous examples. Indeed, the function is identical to the one in step-\/6.


\begin{DoxyCode}
\textcolor{keyword}{template} <\textcolor{keywordtype}{int} dim>
\textcolor{keywordtype}{void} AdvectionProblem<dim>::output\_results(\textcolor{keyword}{const} \textcolor{keywordtype}{unsigned} \textcolor{keywordtype}{int} cycle)\textcolor{keyword}{ const}
\textcolor{keyword}{}\{
    \{
        GridOut grid\_out;
        std::ofstream output(\textcolor{stringliteral}{"grid-"} + std::to\_string(cycle) + \textcolor{stringliteral}{".eps"});
        grid\_out.write\_eps(triangulation, output);
    \}

    \{
        DataOut<dim> data\_out;
        data\_out.attach\_dof\_handler(dof\_handler);
        data\_out.add\_data\_vector(solution, \textcolor{stringliteral}{"solution"});
        data\_out.build\_patches();

        std::ofstream output(\textcolor{stringliteral}{"solution-"} + std::to\_string(cycle) + \textcolor{stringliteral}{".vtk"});
        data\_out.write\_vtk(output);
    \}
\}
\end{DoxyCode}


... as is the main loop (setup -- solve -- refine)


\begin{DoxyCode}
\textcolor{keyword}{template} <\textcolor{keywordtype}{int} dim>
\textcolor{keywordtype}{void} AdvectionProblem<dim>::run()
\{
    \textcolor{keywordflow}{for} (\textcolor{keywordtype}{unsigned} \textcolor{keywordtype}{int} cycle = 0; cycle < 6; ++cycle) \{
        std::cout << \textcolor{stringliteral}{"Cycle "} << cycle << \textcolor{charliteral}{':'} << std::endl;

        \textcolor{keywordflow}{if} (cycle == 0) \{
            GridGenerator::hyper\_cube(triangulation, -1, 1);
            triangulation.refine\_global(4);
        \} \textcolor{keywordflow}{else} \{
            refine\_grid();
        \}


        std::cout << \textcolor{stringliteral}{"   Number of active cells:       "}
                  << triangulation.n\_active\_cells() << std::endl;

        setup\_system();

        std::cout << \textcolor{stringliteral}{"   Number of degrees of freedom: "} << dof\_handler.n\_dofs()
                  << std::endl;

        assemble\_system();
        solve();
        output\_results(cycle);
    \}
\}
\end{DoxyCode}


\label{_GradientEstimationclassimplementation}%
 \subsubsection*{Gradient\+Estimation class implementation}

Now for the implementation of the {\ttfamily Gradient\+Estimation} class. Let us start by defining constructors for the {\ttfamily Estimate\+Scratch\+Data} class used by the {\ttfamily estimate\+\_\+cell()} function\+:


\begin{DoxyCode}
\textcolor{keyword}{template} <\textcolor{keywordtype}{int} dim>
GradientEstimation::EstimateScratchData<dim>::EstimateScratchData(
    \textcolor{keyword}{const} FiniteElement<dim> &fe, \textcolor{keyword}{const} Vector<double> &solution,
    Vector<float> &error\_per\_cell)
    : fe\_midpoint\_value(fe, QMidpoint<dim>(),
                        update\_values | update\_quadrature\_points),
      solution(solution),
      error\_per\_cell(error\_per\_cell)
\{\}


\textcolor{keyword}{template} <\textcolor{keywordtype}{int} dim>
GradientEstimation::EstimateScratchData<dim>::EstimateScratchData(
    \textcolor{keyword}{const} EstimateScratchData &scratch\_data)
    : fe\_midpoint\_value(scratch\_data.fe\_midpoint\_value.get\_fe(),
                        scratch\_data.fe\_midpoint\_value.get\_quadrature(),
                        update\_values | update\_quadrature\_points),
      solution(scratch\_data.solution),
      error\_per\_cell(scratch\_data.error\_per\_cell)
\{\}
\end{DoxyCode}


Next for the implementation of the {\ttfamily Gradient\+Estimation} class. The first function does not much except for delegating work to the other function, but there is a bit of setup at the top.

Before starting with the work, we check that the vector into which the results are written has the right size. Programming mistakes in which one forgets to size arguments correctly at the calling site are quite common. Because the resulting damage from not catching such errors is often subtle (e.\+g., corruption of data somewhere in memory, or non-\/reproducible results), it is well worth the effort to check for such things.


\begin{DoxyCode}
\textcolor{keyword}{template} <\textcolor{keywordtype}{int} dim>
\textcolor{keywordtype}{void} GradientEstimation::estimate(\textcolor{keyword}{const} DoFHandler<dim> &dof\_handler,
                                  \textcolor{keyword}{const} Vector<double> &solution,
                                  Vector<float> &error\_per\_cell)
\{
    Assert(error\_per\_cell.size() ==
               dof\_handler.get\_triangulation().n\_active\_cells(),
           ExcInvalidVectorLength(
               error\_per\_cell.size(),
               dof\_handler.get\_triangulation().n\_active\_cells()));

    WorkStream::run(dof\_handler.begin\_active(), dof\_handler.end(),
                    &GradientEstimation::template estimate\_cell<dim>,
                    std::function<void(const EstimateCopyData &)>(),
                    EstimateScratchData<dim>(dof\_handler.get\_fe(), solution,
                                             error\_per\_cell),
                    EstimateCopyData());
\}
\end{DoxyCode}


Following now the function that actually computes the finite difference approximation to the gradient. The general outline of the function is to first compute the list of active neighbors of the present cell and then compute the quantities described in the introduction for each of the neighbors. The reason for this order is that it is not a one-\/liner to find a given neighbor with locally refined meshes. In principle, an optimized implementation would find neighbors and the quantities depending on them in one step, rather than first building a list of neighbors and in a second step their contributions but we will gladly leave this as an exercise. As discussed before, the worker function passed to Work\+Stream\+::run works on \char`\"{}scratch\char`\"{} objects that keep all temporary objects. This way, we do not need to create and initialize objects that are expensive to initialize within the function that does the work, every time it is called for a given cell. Such an argument is passed as the second argument. The third argument would be a \char`\"{}copy-\/data\char`\"{} object (see threads for more information) but we do not actually use any of these here. Because Work\+Stream\+::run() insists on passing three arguments, we declare this function with three arguments, but simply ignore the last one.

(This is unsatisfactory from an esthetic perspective. It can be avoided, at the cost of some other trickery. If you allow, let us here show how. First, assume that we had declared this function to only take two arguments by omitting the unused last one. Now, Work\+Stream\+::run still wants to call this function with three arguments, so we need to find a way to \char`\"{}forget\char`\"{} the third argument in the call. Simply passing Work\+Stream\+::run the pointer to the function as we do above will not do this -- the compiler will complain that a function declared to have two arguments is called with three arguments. However, we can do this by passing the following as the third argument when calling Work\+Stream\+::run() above\+:  
\begin{DoxyCode}
std::function<void (const typename DoFHandler<dim>::active\_cell\_iterator
&,
                    EstimateScratchData<dim>                  &,
                    EstimateCopyData                          &)>
  (std::bind (&GradientEstimation::template estimate\_cell<dim>,
              std::placeholders::\_1,
              std::placeholders::\_2))
\end{DoxyCode}
  This creates a function object taking three arguments, but when it calls the underlying function object, it simply only uses the first and second argument -- we simply \char`\"{}forget\char`\"{} to use the third argument \+:-\/) In the end, this isn\textquotesingle{}t completely obvious either, and so we didn\textquotesingle{}t implement it, but hey -- it can be done!)

Now for the details\+:


\begin{DoxyCode}
\textcolor{keyword}{template} <\textcolor{keywordtype}{int} dim>
\textcolor{keywordtype}{void} GradientEstimation::estimate\_cell(
    \textcolor{keyword}{const} \textcolor{keyword}{typename} DoFHandler<dim>::active\_cell\_iterator &cell,
    EstimateScratchData<dim> &scratch\_data, \textcolor{keyword}{const} EstimateCopyData &)
\{
\end{DoxyCode}


We need space for the tensor {\ttfamily Y}, which is the sum of outer products of the y-\/vectors.


\begin{DoxyCode}
Tensor<2, dim> Y;
\end{DoxyCode}


Then we allocate a vector to hold iterators to all active neighbors of a cell. We reserve the maximal number of active neighbors in order to avoid later reallocations. Note how this maximal number of active neighbors is computed here.


\begin{DoxyCode}
std::vector<typename DoFHandler<dim>::active\_cell\_iterator>
    active\_neighbors;
active\_neighbors.reserve(GeometryInfo<dim>::faces\_per\_cell *
                         GeometryInfo<dim>::max\_children\_per\_face);
\end{DoxyCode}


First initialize the {\ttfamily F\+E\+Values} object, as well as the {\ttfamily Y} tensor\+:


\begin{DoxyCode}
scratch\_data.fe\_midpoint\_value.reinit(cell);
\end{DoxyCode}


Then allocate the vector that will be the sum over the y-\/vectors times the approximate directional derivative\+:


\begin{DoxyCode}
Tensor<1, dim> projected\_gradient;
\end{DoxyCode}


Now before going on first compute a list of all active neighbors of the present cell. We do so by first looping over all faces and see whether the neighbor there is active, which would be the case if it is on the same level as the present cell or one level coarser (note that a neighbor can only be once coarser than the present cell, as we only allow a maximal difference of one refinement over a face in deal.\+II). Alternatively, the neighbor could be on the same level and be further refined; then we have to find which of its children are next to the present cell and select these (note that if a child of a neighbor of an active cell that is next to this active cell, needs necessarily be active itself, due to the one-\/refinement rule cited above).

Things are slightly different in one space dimension, as there the one-\/refinement rule does not exist\+: neighboring active cells may differ in as many refinement levels as they like. In this case, the computation becomes a little more difficult, but we will explain this below.

Before starting the loop over all neighbors of the present cell, we have to clear the array storing the iterators to the active neighbors, of course.


\begin{DoxyCode}
active\_neighbors.clear();
\textcolor{keywordflow}{for} (\textcolor{keywordtype}{unsigned} \textcolor{keywordtype}{int} face\_no = 0; face\_no < GeometryInfo<dim>::faces\_per\_cell;
     ++face\_no)
    \textcolor{keywordflow}{if} (!cell->at\_boundary(face\_no)) \{
\end{DoxyCode}


First define an abbreviation for the iterator to the face and the neighbor


\begin{DoxyCode}
\textcolor{keyword}{const} \textcolor{keyword}{typename} DoFHandler<dim>::face\_iterator face =
    cell->face(face\_no);
\textcolor{keyword}{const} \textcolor{keyword}{typename} DoFHandler<dim>::cell\_iterator neighbor =
    cell->neighbor(face\_no);
\end{DoxyCode}


Then check whether the neighbor is active. If it is, then it is on the same level or one level coarser (if we are not in 1D), and we are interested in it in any case.


\begin{DoxyCode}
\textcolor{keywordflow}{if} (neighbor->active())
    active\_neighbors.push\_back(neighbor);
\textcolor{keywordflow}{else} \{
\end{DoxyCode}


If the neighbor is not active, then check its children.


\begin{DoxyCode}
\textcolor{keywordflow}{if} (dim == 1) \{
\end{DoxyCode}


To find the child of the neighbor which bounds to the present cell, successively go to its right child if we are left of the present cell (n==0), or go to the left child if we are on the right (n==1), until we find an active cell.


\begin{DoxyCode}
\textcolor{keyword}{typename} DoFHandler<dim>::cell\_iterator neighbor\_child =
    neighbor;
\textcolor{keywordflow}{while} (neighbor\_child->has\_children())
    neighbor\_child =
        neighbor\_child->child(face\_no == 0 ? 1 : 0);
\end{DoxyCode}


As this used some non-\/trivial geometrical intuition, we might want to check whether we did it right, i.\+e. check whether the neighbor of the cell we found is indeed the cell we are presently working on. Checks like this are often useful and have frequently uncovered errors both in algorithms like the line above (where it is simple to involuntarily exchange {\ttfamily n==1} for {\ttfamily n==0} or the like) and in the library (the assumptions underlying the algorithm above could either be wrong, wrongly documented, or are violated due to an error in the library). One could in principle remove such checks after the program works for some time, but it might be a good things to leave it in anyway to check for changes in the library or in the algorithm above.

Note that if this check fails, then this is certainly an error that is irrecoverable and probably qualifies as an internal error. We therefore use a predefined exception class to throw here.


\begin{DoxyCode}
Assert(
    neighbor\_child->neighbor(face\_no == 0 ? 1 : 0) == cell,
    ExcInternalError());
\end{DoxyCode}


If the check succeeded, we push the active neighbor we just found to the stack we keep\+:


\begin{DoxyCode}
    active\_neighbors.push\_back(neighbor\_child);
\} \textcolor{keywordflow}{else}
\end{DoxyCode}


If we are not in 1d, we collect all neighbor children `behind\textquotesingle{} the subfaces of the current face


\begin{DoxyCode}
            \textcolor{keywordflow}{for} (\textcolor{keywordtype}{unsigned} \textcolor{keywordtype}{int} subface\_no = 0;
                 subface\_no < face->n\_children(); ++subface\_no)
                active\_neighbors.push\_back(
                    cell->neighbor\_child\_on\_subface(face\_no,
                                                    subface\_no));
    \}
\}
\end{DoxyCode}


OK, now that we have all the neighbors, lets start the computation on each of them. First we do some preliminaries\+: find out about the center of the present cell and the solution at this point. The latter is obtained as a vector of function values at the quadrature points, of which there are only one, of course. Likewise, the position of the center is the position of the first (and only) quadrature point in real space.


\begin{DoxyCode}
\textcolor{keyword}{const} Point<dim> this\_center =
    scratch\_data.fe\_midpoint\_value.quadrature\_point(0);

std::vector<double> this\_midpoint\_value(1);
scratch\_data.fe\_midpoint\_value.get\_function\_values(scratch\_data.solution,
                                                   this\_midpoint\_value);
\end{DoxyCode}


Now loop over all active neighbors and collect the data we need. Allocate a vector just like {\ttfamily this\+\_\+midpoint\+\_\+value} which we will use to store the value of the solution in the midpoint of the neighbor cell. We allocate it here already, since that way we don\textquotesingle{}t have to allocate memory repeatedly in each iteration of this inner loop (memory allocation is a rather expensive operation)\+:


\begin{DoxyCode}
std::vector<double> neighbor\_midpoint\_value(1);
\textcolor{keyword}{typename} std::vector<typename DoFHandler<dim>::active\_cell\_iterator>::
    const\_iterator neighbor\_ptr = active\_neighbors.begin();
\textcolor{keywordflow}{for} (; neighbor\_ptr != active\_neighbors.end(); ++neighbor\_ptr) \{
\end{DoxyCode}


First define an abbreviation for the iterator to the active neighbor cell\+:


\begin{DoxyCode}
\textcolor{keyword}{const} \textcolor{keyword}{typename} DoFHandler<dim>::active\_cell\_iterator neighbor =
    *neighbor\_ptr;
\end{DoxyCode}


Then get the center of the neighbor cell and the value of the finite element function thereon. Note that for this information we have to reinitialize the {\ttfamily F\+E\+Values} object for the neighbor cell.


\begin{DoxyCode}
scratch\_data.fe\_midpoint\_value.reinit(neighbor);
\textcolor{keyword}{const} Point<dim> neighbor\_center =
    scratch\_data.fe\_midpoint\_value.quadrature\_point(0);

scratch\_data.fe\_midpoint\_value.get\_function\_values(
    scratch\_data.solution, neighbor\_midpoint\_value);
\end{DoxyCode}


Compute the vector {\ttfamily y} connecting the centers of the two cells. Note that as opposed to the introduction, we denote by {\ttfamily y} the normalized difference vector, as this is the quantity used everywhere in the computations.


\begin{DoxyCode}
Tensor<1, dim> y = neighbor\_center - this\_center;
\textcolor{keyword}{const} \textcolor{keywordtype}{double} distance = y.norm();
y /= distance;
\end{DoxyCode}


Then add up the contribution of this cell to the Y matrix...


\begin{DoxyCode}
\textcolor{keywordflow}{for} (\textcolor{keywordtype}{unsigned} \textcolor{keywordtype}{int} i = 0; i < dim; ++i)
    \textcolor{keywordflow}{for} (\textcolor{keywordtype}{unsigned} \textcolor{keywordtype}{int} j = 0; j < dim; ++j) Y[i][j] += y[i] * y[j];
\end{DoxyCode}


... and update the sum of difference quotients\+:


\begin{DoxyCode}
    projected\_gradient +=
        (neighbor\_midpoint\_value[0] - this\_midpoint\_value[0]) / distance *
        y;
\}
\end{DoxyCode}


If now, after collecting all the information from the neighbors, we can determine an approximation of the gradient for the present cell, then we need to have passed over vectors {\ttfamily y} which span the whole space, otherwise we would not have all components of the gradient. This is indicated by the invertibility of the matrix.

If the matrix should not be invertible, this means that the present cell had an insufficient number of active neighbors. In contrast to all previous cases, where we raised exceptions, this is, however, not a programming error\+: it is a runtime error that can happen in optimized mode even if it ran well in debug mode, so it is reasonable to try to catch this error also in optimized mode. For this case, there is the {\ttfamily Assert\+Throw} macro\+: it checks the condition like the {\ttfamily Assert} macro, but not only in debug mode; it then outputs an error message, but instead of terminating the program as in the case of the {\ttfamily Assert} macro, the exception is thrown using the {\ttfamily throw} command of C++. This way, one has the possibility to catch this error and take reasonable counter actions. One such measure would be to refine the grid globally, as the case of insufficient directions can not occur if every cell of the initial grid has been refined at least once.


\begin{DoxyCode}
AssertThrow(determinant(Y) != 0, ExcInsufficientDirections());
\end{DoxyCode}


If, on the other hand the matrix is invertible, then invert it, multiply the other quantity with it and compute the estimated error using this quantity and the right powers of the mesh width\+:


\begin{DoxyCode}
\textcolor{keyword}{const} Tensor<2, dim> Y\_inverse = invert(Y);

Tensor<1, dim> gradient = Y\_inverse * projected\_gradient;
\end{DoxyCode}


The last part of this function is the one where we write into the element of the output vector what we have just computed. The address of this vector has been stored in the scratch data object, and all we have to do is know how to get at the correct element inside this vector -- but we can ask the cell we\textquotesingle{}re on the how-\/manyth active cell it is for this\+:


\begin{DoxyCode}
    scratch\_data.error\_per\_cell(cell->active\_cell\_index()) =
        (std::pow(cell->diameter(), 1 + 1.0 * dim / 2) *
         std::sqrt(gradient.norm\_square()));
\}
\}  \textcolor{comment}{// namespace Step9}
\end{DoxyCode}


\label{_Mainfunction}%
 \subsubsection*{Main function}

The {\ttfamily main} function is similar to the previous examples. The main difference is that we use Multithread\+Info to set the maximum number of threads (see Parallel computing with multiple processors accessing shared memory" documentation module for more explanation). The number of threads used is the minimum of the environment variable D\+E\+A\+L\+\_\+\+I\+I\+\_\+\+N\+U\+M\+\_\+\+T\+H\+R\+E\+A\+DS and the parameter of {\ttfamily set\+\_\+thread\+\_\+limit}. If no value is given to {\ttfamily set\+\_\+thread\+\_\+limit}, the default value from the Intel Threading Building Blocks (T\+BB) library is used. If the call to {\ttfamily set\+\_\+thread\+\_\+limit} is omitted, the number of threads will be chosen by T\+BB indepently of D\+E\+A\+L\+\_\+\+I\+I\+\_\+\+N\+U\+M\+\_\+\+T\+H\+R\+E\+A\+DS.


\begin{DoxyCode}
\textcolor{keywordtype}{int} main()
\{
    \textcolor{keywordflow}{try} \{
        dealii::MultithreadInfo::set\_thread\_limit();

        Step9::AdvectionProblem<2> advection\_problem\_2d;
        advection\_problem\_2d.run();
    \} \textcolor{keywordflow}{catch} (std::exception &exc) \{
        std::cerr << std::endl
                  << std::endl
                  << \textcolor{stringliteral}{"----------------------------------------------------"}
                  << std::endl;
        std::cerr << \textcolor{stringliteral}{"Exception on processing: "} << std::endl
                  << exc.what() << std::endl
                  << \textcolor{stringliteral}{"Aborting!"} << std::endl
                  << \textcolor{stringliteral}{"----------------------------------------------------"}
                  << std::endl;
        \textcolor{keywordflow}{return} 1;
    \} \textcolor{keywordflow}{catch} (...) \{
        std::cerr << std::endl
                  << std::endl
                  << \textcolor{stringliteral}{"----------------------------------------------------"}
                  << std::endl;
        std::cerr << \textcolor{stringliteral}{"Unknown exception!"} << std::endl
                  << \textcolor{stringliteral}{"Aborting!"} << std::endl
                  << \textcolor{stringliteral}{"----------------------------------------------------"}
                  << std::endl;
        \textcolor{keywordflow}{return} 1;
    \}

    \textcolor{keywordflow}{return} 0;
\}
\end{DoxyCode}
 \label{_Results}%
\section*{Results}

\label{_Commentsaboutprogramminganddebugging}%
\paragraph*{Comments about programming and debugging }

\label{_PlainProg}%
 \section*{The plain program}


\begin{DoxyCodeInclude}
\textcolor{comment}{/*******************************<GINKGO LICENSE>******************************}
\textcolor{comment}{Copyright (c) 2017-2019, the Ginkgo authors}
\textcolor{comment}{All rights reserved.}
\textcolor{comment}{}
\textcolor{comment}{Redistribution and use in source and binary forms, with or without}
\textcolor{comment}{modification, are permitted provided that the following conditions}
\textcolor{comment}{are met:}
\textcolor{comment}{}
\textcolor{comment}{1. Redistributions of source code must retain the above copyright}
\textcolor{comment}{notice, this list of conditions and the following disclaimer.}
\textcolor{comment}{}
\textcolor{comment}{2. Redistributions in binary form must reproduce the above copyright}
\textcolor{comment}{notice, this list of conditions and the following disclaimer in the}
\textcolor{comment}{documentation and/or other materials provided with the distribution.}
\textcolor{comment}{}
\textcolor{comment}{3. Neither the name of the copyright holder nor the names of its}
\textcolor{comment}{contributors may be used to endorse or promote products derived from}
\textcolor{comment}{this software without specific prior written permission.}
\textcolor{comment}{}
\textcolor{comment}{THIS SOFTWARE IS PROVIDED BY THE COPYRIGHT HOLDERS AND CONTRIBUTORS "AS}
\textcolor{comment}{IS" AND ANY EXPRESS OR IMPLIED WARRANTIES, INCLUDING, BUT NOT LIMITED}
\textcolor{comment}{TO, THE IMPLIED WARRANTIES OF MERCHANTABILITY AND FITNESS FOR A}
\textcolor{comment}{PARTICULAR PURPOSE ARE DISCLAIMED. IN NO EVENT SHALL THE COPYRIGHT}
\textcolor{comment}{HOLDER OR CONTRIBUTORS BE LIABLE FOR ANY DIRECT, INDIRECT, INCIDENTAL,}
\textcolor{comment}{SPECIAL, EXEMPLARY, OR CONSEQUENTIAL DAMAGES (INCLUDING, BUT NOT}
\textcolor{comment}{LIMITED TO, PROCUREMENT OF SUBSTITUTE GOODS OR SERVICES; LOSS OF USE,}
\textcolor{comment}{DATA, OR PROFITS; OR BUSINESS INTERRUPTION) HOWEVER CAUSED AND ON ANY}
\textcolor{comment}{THEORY OF LIABILITY, WHETHER IN CONTRACT, STRICT LIABILITY, OR TORT}
\textcolor{comment}{(INCLUDING NEGLIGENCE OR OTHERWISE) ARISING IN ANY WAY OUT OF THE USE}
\textcolor{comment}{OF THIS SOFTWARE, EVEN IF ADVISED OF THE POSSIBILITY OF SUCH DAMAGE.}
\textcolor{comment}{******************************<GINKGO LICENSE>*******************************/}

\textcolor{comment}{/* ---------------------------------------------------------------------}
\textcolor{comment}{ *}
\textcolor{comment}{ * Copyright (C) 2000 - 2018 by the deal.II authors}
\textcolor{comment}{ *}
\textcolor{comment}{ * This file is part of the deal.II library.}
\textcolor{comment}{ *}
\textcolor{comment}{ * The deal.II library is free software; you can use it, redistribute}
\textcolor{comment}{ * it, and/or modify it under the terms of the GNU Lesser General}
\textcolor{comment}{ * Public License as published by the Free Software Foundation; either}
\textcolor{comment}{ * version 2.1 of the License, or (at your option) any later version.}
\textcolor{comment}{ * The full text of the license can be found in the file LICENSE at}
\textcolor{comment}{ * the top level of the deal.II distribution.}
\textcolor{comment}{ *}
\textcolor{comment}{ * ---------------------------------------------------------------------}
\textcolor{comment}{}
\textcolor{comment}{ *}
\textcolor{comment}{ * Author: Wolfgang Bangerth, University of Heidelberg, 2000}
\textcolor{comment}{ */}

\textcolor{comment}{/* ---------------------------------------------------------------------}
\textcolor{comment}{ *}
\textcolor{comment}{ * This file has been taken verbatim from the deal.ii (version 9.0)}
\textcolor{comment}{ * examples directory and modified.}
\textcolor{comment}{ *}
\textcolor{comment}{ * This example aims to demonstrate the ease with which Ginkgo can}
\textcolor{comment}{ * be interfaced with other libraries. The only modification/ addition}
\textcolor{comment}{ * has been to the AdvectionProblem::solve () function.}
\textcolor{comment}{ *}
\textcolor{comment}{ */}

\textcolor{preprocessor}{#include <deal.II/base/function.h>}
\textcolor{preprocessor}{#include <deal.II/base/logstream.h>}
\textcolor{preprocessor}{#include <deal.II/base/quadrature\_lib.h>}
\textcolor{preprocessor}{#include <deal.II/dofs/dof\_accessor.h>}
\textcolor{preprocessor}{#include <deal.II/dofs/dof\_handler.h>}
\textcolor{preprocessor}{#include <deal.II/dofs/dof\_tools.h>}
\textcolor{preprocessor}{#include <deal.II/fe/fe\_q.h>}
\textcolor{preprocessor}{#include <deal.II/fe/fe\_values.h>}
\textcolor{preprocessor}{#include <deal.II/grid/grid\_generator.h>}
\textcolor{preprocessor}{#include <deal.II/grid/grid\_out.h>}
\textcolor{preprocessor}{#include <deal.II/grid/grid\_refinement.h>}
\textcolor{preprocessor}{#include <deal.II/grid/tria.h>}
\textcolor{preprocessor}{#include <deal.II/grid/tria\_accessor.h>}
\textcolor{preprocessor}{#include <deal.II/grid/tria\_iterator.h>}
\textcolor{preprocessor}{#include <deal.II/lac/constraint\_matrix.h>}
\textcolor{preprocessor}{#include <deal.II/lac/dynamic\_sparsity\_pattern.h>}
\textcolor{preprocessor}{#include <deal.II/lac/full\_matrix.h>}
\textcolor{preprocessor}{#include <deal.II/lac/precondition.h>}
\textcolor{preprocessor}{#include <deal.II/lac/solver\_bicgstab.h>}
\textcolor{preprocessor}{#include <deal.II/lac/sparse\_matrix.h>}
\textcolor{preprocessor}{#include <deal.II/lac/vector.h>}
\textcolor{preprocessor}{#include <deal.II/numerics/data\_out.h>}
\textcolor{preprocessor}{#include <deal.II/numerics/matrix\_tools.h>}
\textcolor{preprocessor}{#include <deal.II/numerics/vector\_tools.h>}

\textcolor{preprocessor}{#include <deal.II/base/multithread\_info.h>}
\textcolor{preprocessor}{#include <deal.II/base/work\_stream.h>}

\textcolor{preprocessor}{#include <deal.II/base/tensor\_function.h>}

\textcolor{preprocessor}{#include <deal.II/numerics/error\_estimator.h>}

\textcolor{preprocessor}{#include <ginkgo/ginkgo.hpp>}

\textcolor{preprocessor}{#include <fstream>}
\textcolor{preprocessor}{#include <iostream>}


\textcolor{keyword}{namespace }Step9 \{
\textcolor{keyword}{using namespace }dealii;


\textcolor{keyword}{template} <\textcolor{keywordtype}{int} dim>
\textcolor{keyword}{class }AdvectionProblem \{
\textcolor{keyword}{public}:
    AdvectionProblem();
    ~AdvectionProblem();
    \textcolor{keywordtype}{void} run();

\textcolor{keyword}{private}:
    \textcolor{keywordtype}{void} setup\_system();

    \textcolor{keyword}{struct }AssemblyScratchData \{
        AssemblyScratchData(\textcolor{keyword}{const} FiniteElement<dim> &fe);
        AssemblyScratchData(\textcolor{keyword}{const} AssemblyScratchData &scratch\_data);

        FEValues<dim> fe\_values;
        FEFaceValues<dim> fe\_face\_values;
    \};

    \textcolor{keyword}{struct }AssemblyCopyData \{
        FullMatrix<double> cell\_matrix;
        Vector<double> cell\_rhs;
        std::vector<types::global\_dof\_index> local\_dof\_indices;
    \};

    \textcolor{keywordtype}{void} assemble\_system();
    \textcolor{keywordtype}{void} local\_assemble\_system(
        \textcolor{keyword}{const} \textcolor{keyword}{typename} DoFHandler<dim>::active\_cell\_iterator &cell,
        AssemblyScratchData &scratch, AssemblyCopyData &copy\_data);
    \textcolor{keywordtype}{void} copy\_local\_to\_global(\textcolor{keyword}{const} AssemblyCopyData &copy\_data);


    \textcolor{keywordtype}{void} solve();
    \textcolor{keywordtype}{void} refine\_grid();
    \textcolor{keywordtype}{void} output\_results(\textcolor{keyword}{const} \textcolor{keywordtype}{unsigned} \textcolor{keywordtype}{int} cycle) \textcolor{keyword}{const};

    Triangulation<dim> triangulation;
    DoFHandler<dim> dof\_handler;

    FE\_Q<dim> fe;

    ConstraintMatrix hanging\_node\_constraints;

    SparsityPattern sparsity\_pattern;
    SparseMatrix<double> system\_matrix;

    Vector<double> solution;
    Vector<double> system\_rhs;
\};



\textcolor{keyword}{template} <\textcolor{keywordtype}{int} dim>
\textcolor{keyword}{class }AdvectionField : \textcolor{keyword}{public} TensorFunction<1, dim> \{
\textcolor{keyword}{public}:
    AdvectionField() : TensorFunction<1, dim>() \{\}

    \textcolor{keyword}{virtual} Tensor<1, dim> value(\textcolor{keyword}{const} Point<dim> &p) \textcolor{keyword}{const};

    \textcolor{keyword}{virtual} \textcolor{keywordtype}{void} value\_list(\textcolor{keyword}{const} std::vector<Point<dim>> &points,
                            std::vector<Tensor<1, dim>> &values) \textcolor{keyword}{const};

    DeclException2(ExcDimensionMismatch, \textcolor{keywordtype}{unsigned} \textcolor{keywordtype}{int}, \textcolor{keywordtype}{unsigned} \textcolor{keywordtype}{int},
                   << \textcolor{stringliteral}{"The vector has size "} << arg1 << \textcolor{stringliteral}{" but should have "}
                   << arg2 << \textcolor{stringliteral}{" elements."});
\};


\textcolor{keyword}{template} <\textcolor{keywordtype}{int} dim>
Tensor<1, dim> AdvectionField<dim>::value(\textcolor{keyword}{const} Point<dim> &p)\textcolor{keyword}{ const}
\textcolor{keyword}{}\{
    Point<dim> value;
    value[0] = 2;
    \textcolor{keywordflow}{for} (\textcolor{keywordtype}{unsigned} \textcolor{keywordtype}{int} i = 1; i < dim; ++i)
        value[i] = 1 + 0.8 * std::sin(8 * numbers::PI * p[0]);

    \textcolor{keywordflow}{return} value;
\}


\textcolor{keyword}{template} <\textcolor{keywordtype}{int} dim>
\textcolor{keywordtype}{void} AdvectionField<dim>::value\_list(\textcolor{keyword}{const} std::vector<Point<dim>> &points,
                                     std::vector<Tensor<1, dim>> &values)\textcolor{keyword}{ const}
\textcolor{keyword}{}\{
    Assert(values.size() == points.size(),
           ExcDimensionMismatch(values.size(), points.size()));

    \textcolor{keywordflow}{for} (\textcolor{keywordtype}{unsigned} \textcolor{keywordtype}{int} i = 0; i < points.size(); ++i)
        values[i] = AdvectionField<dim>::value(points[i]);
\}


\textcolor{keyword}{template} <\textcolor{keywordtype}{int} dim>
\textcolor{keyword}{class }RightHandSide : \textcolor{keyword}{public} Function<dim> \{
\textcolor{keyword}{public}:
    RightHandSide() : Function<dim>() \{\}

    \textcolor{keyword}{virtual} \textcolor{keywordtype}{double} value(\textcolor{keyword}{const} Point<dim> &p,
                         \textcolor{keyword}{const} \textcolor{keywordtype}{unsigned} \textcolor{keywordtype}{int} component = 0) \textcolor{keyword}{const};

    \textcolor{keyword}{virtual} \textcolor{keywordtype}{void} value\_list(\textcolor{keyword}{const} std::vector<Point<dim>> &points,
                            std::vector<double> &values,
                            \textcolor{keyword}{const} \textcolor{keywordtype}{unsigned} \textcolor{keywordtype}{int} component = 0) \textcolor{keyword}{const};

\textcolor{keyword}{private}:
    \textcolor{keyword}{static} \textcolor{keyword}{const} Point<dim> center\_point;
\};


\textcolor{keyword}{template} <>
\textcolor{keyword}{const} Point<1> RightHandSide<1>::center\_point = Point<1>(-0.75);

\textcolor{keyword}{template} <>
\textcolor{keyword}{const} Point<2> RightHandSide<2>::center\_point = Point<2>(-0.75, -0.75);

\textcolor{keyword}{template} <>
\textcolor{keyword}{const} Point<3> RightHandSide<3>::center\_point = Point<3>(-0.75, -0.75, -0.75);


\textcolor{keyword}{template} <\textcolor{keywordtype}{int} dim>
\textcolor{keywordtype}{double} RightHandSide<dim>::value(\textcolor{keyword}{const} Point<dim> &p,
                                 \textcolor{keyword}{const} \textcolor{keywordtype}{unsigned} \textcolor{keywordtype}{int} component)\textcolor{keyword}{ const}
\textcolor{keyword}{}\{
    (void)component;
    Assert(component == 0, ExcIndexRange(component, 0, 1));
    \textcolor{keyword}{const} \textcolor{keywordtype}{double} diameter = 0.1;
    \textcolor{keywordflow}{return} ((p - center\_point).norm\_square() < diameter * diameter
                ? .1 / std::pow(diameter, dim)
                : 0);
\}


\textcolor{keyword}{template} <\textcolor{keywordtype}{int} dim>
\textcolor{keywordtype}{void} RightHandSide<dim>::value\_list(\textcolor{keyword}{const} std::vector<Point<dim>> &points,
                                    std::vector<double> &values,
                                    \textcolor{keyword}{const} \textcolor{keywordtype}{unsigned} \textcolor{keywordtype}{int} component)\textcolor{keyword}{ const}
\textcolor{keyword}{}\{
    Assert(values.size() == points.size(),
           ExcDimensionMismatch(values.size(), points.size()));

    \textcolor{keywordflow}{for} (\textcolor{keywordtype}{unsigned} \textcolor{keywordtype}{int} i = 0; i < points.size(); ++i)
        values[i] = RightHandSide<dim>::value(points[i], component);
\}


\textcolor{keyword}{template} <\textcolor{keywordtype}{int} dim>
\textcolor{keyword}{class }BoundaryValues : \textcolor{keyword}{public} Function<dim> \{
\textcolor{keyword}{public}:
    BoundaryValues() : Function<dim>() \{\}

    \textcolor{keyword}{virtual} \textcolor{keywordtype}{double} value(\textcolor{keyword}{const} Point<dim> &p,
                         \textcolor{keyword}{const} \textcolor{keywordtype}{unsigned} \textcolor{keywordtype}{int} component = 0) \textcolor{keyword}{const};

    \textcolor{keyword}{virtual} \textcolor{keywordtype}{void} value\_list(\textcolor{keyword}{const} std::vector<Point<dim>> &points,
                            std::vector<double> &values,
                            \textcolor{keyword}{const} \textcolor{keywordtype}{unsigned} \textcolor{keywordtype}{int} component = 0) \textcolor{keyword}{const};
\};


\textcolor{keyword}{template} <\textcolor{keywordtype}{int} dim>
\textcolor{keywordtype}{double} BoundaryValues<dim>::value(\textcolor{keyword}{const} Point<dim> &p,
                                  \textcolor{keyword}{const} \textcolor{keywordtype}{unsigned} \textcolor{keywordtype}{int} component)\textcolor{keyword}{ const}
\textcolor{keyword}{}\{
    (void)component;
    Assert(component == 0, ExcIndexRange(component, 0, 1));

    \textcolor{keyword}{const} \textcolor{keywordtype}{double} sine\_term =
        std::sin(16 * numbers::PI * std::sqrt(p.norm\_square()));
    \textcolor{keyword}{const} \textcolor{keywordtype}{double} weight = std::exp(-5 * p.norm\_square()) / std::exp(-5.);
    \textcolor{keywordflow}{return} sine\_term * weight;
\}


\textcolor{keyword}{template} <\textcolor{keywordtype}{int} dim>
\textcolor{keywordtype}{void} BoundaryValues<dim>::value\_list(\textcolor{keyword}{const} std::vector<Point<dim>> &points,
                                     std::vector<double> &values,
                                     \textcolor{keyword}{const} \textcolor{keywordtype}{unsigned} \textcolor{keywordtype}{int} component)\textcolor{keyword}{ const}
\textcolor{keyword}{}\{
    Assert(values.size() == points.size(),
           ExcDimensionMismatch(values.size(), points.size()));

    \textcolor{keywordflow}{for} (\textcolor{keywordtype}{unsigned} \textcolor{keywordtype}{int} i = 0; i < points.size(); ++i)
        values[i] = BoundaryValues<dim>::value(points[i], component);
\}



\textcolor{keyword}{class }GradientEstimation \{
\textcolor{keyword}{public}:
    \textcolor{keyword}{template} <\textcolor{keywordtype}{int} dim>
    \textcolor{keyword}{static} \textcolor{keywordtype}{void} estimate(\textcolor{keyword}{const} DoFHandler<dim> &dof,
                         \textcolor{keyword}{const} Vector<double> &solution,
                         Vector<float> &error\_per\_cell);

    DeclException2(ExcInvalidVectorLength, \textcolor{keywordtype}{int}, \textcolor{keywordtype}{int},
                   << \textcolor{stringliteral}{"Vector has length "} << arg1 << \textcolor{stringliteral}{", but should have "}
                   << arg2);
    DeclException0(ExcInsufficientDirections);

\textcolor{keyword}{private}:
    \textcolor{keyword}{template} <\textcolor{keywordtype}{int} dim>
    \textcolor{keyword}{struct }EstimateScratchData \{
        EstimateScratchData(\textcolor{keyword}{const} FiniteElement<dim> &fe,
                            \textcolor{keyword}{const} Vector<double> &solution,
                            Vector<float> &error\_per\_cell);
        EstimateScratchData(\textcolor{keyword}{const} EstimateScratchData &data);

        FEValues<dim> fe\_midpoint\_value;
        \textcolor{keyword}{const} Vector<double> &solution;
        Vector<float> &error\_per\_cell;
    \};

    \textcolor{keyword}{struct }EstimateCopyData \{\};

    \textcolor{keyword}{template} <\textcolor{keywordtype}{int} dim>
    \textcolor{keyword}{static} \textcolor{keywordtype}{void} estimate\_cell(
        \textcolor{keyword}{const} \textcolor{keyword}{typename} DoFHandler<dim>::active\_cell\_iterator &cell,
        EstimateScratchData<dim> &scratch\_data,
        \textcolor{keyword}{const} EstimateCopyData &copy\_data);
\};




\textcolor{keyword}{template} <\textcolor{keywordtype}{int} dim>
AdvectionProblem<dim>::AdvectionProblem() : dof\_handler(triangulation), fe(1)
\{\}


\textcolor{keyword}{template} <\textcolor{keywordtype}{int} dim>
AdvectionProblem<dim>::~AdvectionProblem()
\{
    dof\_handler.clear();
\}


\textcolor{keyword}{template} <\textcolor{keywordtype}{int} dim>
\textcolor{keywordtype}{void} AdvectionProblem<dim>::setup\_system()
\{
    dof\_handler.distribute\_dofs(fe);
    hanging\_node\_constraints.clear();
    DoFTools::make\_hanging\_node\_constraints(dof\_handler,
                                            hanging\_node\_constraints);
    hanging\_node\_constraints.close();

    DynamicSparsityPattern dsp(dof\_handler.n\_dofs(), dof\_handler.n\_dofs());
    DoFTools::make\_sparsity\_pattern(dof\_handler, dsp, hanging\_node\_constraints,
                                    \textcolor{comment}{/*keep\_constrained\_dofs = */} \textcolor{keyword}{true});
    sparsity\_pattern.copy\_from(dsp);

    system\_matrix.reinit(sparsity\_pattern);

    solution.reinit(dof\_handler.n\_dofs());
    system\_rhs.reinit(dof\_handler.n\_dofs());
\}


\textcolor{keyword}{template} <\textcolor{keywordtype}{int} dim>
\textcolor{keywordtype}{void} AdvectionProblem<dim>::assemble\_system()
\{
    WorkStream::run(dof\_handler.begin\_active(), dof\_handler.end(), *\textcolor{keyword}{this},
                    &AdvectionProblem::local\_assemble\_system,
                    &AdvectionProblem::copy\_local\_to\_global,
                    AssemblyScratchData(fe), AssemblyCopyData());


    hanging\_node\_constraints.condense(system\_matrix);
    hanging\_node\_constraints.condense(system\_rhs);
\}


\textcolor{keyword}{template} <\textcolor{keywordtype}{int} dim>
AdvectionProblem<dim>::AssemblyScratchData::AssemblyScratchData(
    \textcolor{keyword}{const} FiniteElement<dim> &fe)
    : fe\_values(fe, QGauss<dim>(2),
                update\_values | update\_gradients | update\_quadrature\_points |
                    update\_JxW\_values),
      fe\_face\_values(fe, QGauss<dim - 1>(2),
                     update\_values | update\_quadrature\_points |
                         update\_JxW\_values | update\_normal\_vectors)
\{\}


\textcolor{keyword}{template} <\textcolor{keywordtype}{int} dim>
AdvectionProblem<dim>::AssemblyScratchData::AssemblyScratchData(
    \textcolor{keyword}{const} AssemblyScratchData &scratch\_data)
    : fe\_values(scratch\_data.fe\_values.get\_fe(),
                scratch\_data.fe\_values.get\_quadrature(),
                update\_values | update\_gradients | update\_quadrature\_points |
                    update\_JxW\_values),
      fe\_face\_values(scratch\_data.fe\_face\_values.get\_fe(),
                     scratch\_data.fe\_face\_values.get\_quadrature(),
                     update\_values | update\_quadrature\_points |
                         update\_JxW\_values | update\_normal\_vectors)
\{\}


\textcolor{keyword}{template} <\textcolor{keywordtype}{int} dim>
\textcolor{keywordtype}{void} AdvectionProblem<dim>::local\_assemble\_system(
    \textcolor{keyword}{const} \textcolor{keyword}{typename} DoFHandler<dim>::active\_cell\_iterator &cell,
    AssemblyScratchData &scratch\_data, AssemblyCopyData &copy\_data)
\{
    \textcolor{keyword}{const} AdvectionField<dim> advection\_field;
    \textcolor{keyword}{const} RightHandSide<dim> right\_hand\_side;
    \textcolor{keyword}{const} BoundaryValues<dim> boundary\_values;

    \textcolor{keyword}{const} \textcolor{keywordtype}{unsigned} \textcolor{keywordtype}{int} dofs\_per\_cell = fe.dofs\_per\_cell;
    \textcolor{keyword}{const} \textcolor{keywordtype}{unsigned} \textcolor{keywordtype}{int} n\_q\_points =
        scratch\_data.fe\_values.get\_quadrature().size();
    \textcolor{keyword}{const} \textcolor{keywordtype}{unsigned} \textcolor{keywordtype}{int} n\_face\_q\_points =
        scratch\_data.fe\_face\_values.get\_quadrature().size();

    copy\_data.cell\_matrix.reinit(dofs\_per\_cell, dofs\_per\_cell);
    copy\_data.cell\_rhs.reinit(dofs\_per\_cell);

    copy\_data.local\_dof\_indices.resize(dofs\_per\_cell);

    std::vector<double> rhs\_values(n\_q\_points);
    std::vector<Tensor<1, dim>> advection\_directions(n\_q\_points);
    std::vector<double> face\_boundary\_values(n\_face\_q\_points);
    std::vector<Tensor<1, dim>> face\_advection\_directions(n\_face\_q\_points);


    scratch\_data.fe\_values.reinit(cell);

    advection\_field.value\_list(scratch\_data.fe\_values.get\_quadrature\_points(),
                               advection\_directions);
    right\_hand\_side.value\_list(scratch\_data.fe\_values.get\_quadrature\_points(),
                               rhs\_values);

    \textcolor{keyword}{const} \textcolor{keywordtype}{double} delta = 0.1 * cell->diameter();

    \textcolor{keywordflow}{for} (\textcolor{keywordtype}{unsigned} \textcolor{keywordtype}{int} q\_point = 0; q\_point < n\_q\_points; ++q\_point)
        \textcolor{keywordflow}{for} (\textcolor{keywordtype}{unsigned} \textcolor{keywordtype}{int} i = 0; i < dofs\_per\_cell; ++i) \{
            \textcolor{keywordflow}{for} (\textcolor{keywordtype}{unsigned} \textcolor{keywordtype}{int} j = 0; j < dofs\_per\_cell; ++j)
                copy\_data.cell\_matrix(i, j) +=
                    ((advection\_directions[q\_point] *
                      scratch\_data.fe\_values.shape\_grad(j, q\_point) *
                      (scratch\_data.fe\_values.shape\_value(i, q\_point) +
                       delta *
                           (advection\_directions[q\_point] *
                            scratch\_data.fe\_values.shape\_grad(i, q\_point)))) *
                     scratch\_data.fe\_values.JxW(q\_point));

            copy\_data.cell\_rhs(i) +=
                ((scratch\_data.fe\_values.shape\_value(i, q\_point) +
                  delta * (advection\_directions[q\_point] *
                           scratch\_data.fe\_values.shape\_grad(i, q\_point))) *
                 rhs\_values[q\_point] * scratch\_data.fe\_values.JxW(q\_point));
        \}

    \textcolor{keywordflow}{for} (\textcolor{keywordtype}{unsigned} \textcolor{keywordtype}{int} face = 0; face < GeometryInfo<dim>::faces\_per\_cell;
         ++face)
        \textcolor{keywordflow}{if} (cell->face(face)->at\_boundary()) \{
            scratch\_data.fe\_face\_values.reinit(cell, face);

            boundary\_values.value\_list(
                scratch\_data.fe\_face\_values.get\_quadrature\_points(),
                face\_boundary\_values);
            advection\_field.value\_list(
                scratch\_data.fe\_face\_values.get\_quadrature\_points(),
                face\_advection\_directions);

            \textcolor{keywordflow}{for} (\textcolor{keywordtype}{unsigned} \textcolor{keywordtype}{int} q\_point = 0; q\_point < n\_face\_q\_points; ++q\_point)
                \textcolor{keywordflow}{if} (scratch\_data.fe\_face\_values.normal\_vector(q\_point) *
                        face\_advection\_directions[q\_point] <
                    0)
                    \textcolor{keywordflow}{for} (\textcolor{keywordtype}{unsigned} \textcolor{keywordtype}{int} i = 0; i < dofs\_per\_cell; ++i) \{
                        \textcolor{keywordflow}{for} (\textcolor{keywordtype}{unsigned} \textcolor{keywordtype}{int} j = 0; j < dofs\_per\_cell; ++j)
                            copy\_data.cell\_matrix(i, j) -=
                                (face\_advection\_directions[q\_point] *
                                 scratch\_data.fe\_face\_values.normal\_vector(
                                     q\_point) *
                                 scratch\_data.fe\_face\_values.shape\_value(
                                     i, q\_point) *
                                 scratch\_data.fe\_face\_values.shape\_value(
                                     j, q\_point) *
                                 scratch\_data.fe\_face\_values.JxW(q\_point));

                        copy\_data.cell\_rhs(i) -=
                            (face\_advection\_directions[q\_point] *
                             scratch\_data.fe\_face\_values.normal\_vector(
                                 q\_point) *
                             face\_boundary\_values[q\_point] *
                             scratch\_data.fe\_face\_values.shape\_value(i,
                                                                     q\_point) *
                             scratch\_data.fe\_face\_values.JxW(q\_point));
                    \}
        \}


    cell->get\_dof\_indices(copy\_data.local\_dof\_indices);
\}


\textcolor{keyword}{template} <\textcolor{keywordtype}{int} dim>
\textcolor{keywordtype}{void} AdvectionProblem<dim>::copy\_local\_to\_global(
    \textcolor{keyword}{const} AssemblyCopyData &copy\_data)
\{
    \textcolor{keywordflow}{for} (\textcolor{keywordtype}{unsigned} \textcolor{keywordtype}{int} i = 0; i < copy\_data.local\_dof\_indices.size(); ++i) \{
        \textcolor{keywordflow}{for} (\textcolor{keywordtype}{unsigned} \textcolor{keywordtype}{int} j = 0; j < copy\_data.local\_dof\_indices.size(); ++j)
            system\_matrix.add(copy\_data.local\_dof\_indices[i],
                              copy\_data.local\_dof\_indices[j],
                              copy\_data.cell\_matrix(i, j));

        system\_rhs(copy\_data.local\_dof\_indices[i]) += copy\_data.cell\_rhs(i);
    \}
\}

\textcolor{keyword}{template} <\textcolor{keywordtype}{int} dim>
\textcolor{keywordtype}{void} AdvectionProblem<dim>::solve()
\{
    Assert(system\_matrix.m() == system\_matrix.n(), ExcNotQuadratic());
    \textcolor{keyword}{auto} num\_rows = system\_matrix.m();

    std::vector<double> rhs(num\_rows);
    std::copy(system\_rhs.begin(), system\_rhs.begin() + num\_rows, rhs.begin());

    \textcolor{keyword}{using} vec = \hyperlink{classgko_1_1matrix_1_1Dense}{gko::matrix::Dense<>};
    \textcolor{keyword}{using} mtx = \hyperlink{classgko_1_1matrix_1_1Csr}{gko::matrix::Csr<>};
    \textcolor{keyword}{using} bicgstab = \hyperlink{classgko_1_1solver_1_1Bicgstab}{gko::solver::Bicgstab<>};
    \textcolor{keyword}{using} bj = \hyperlink{classgko_1_1preconditioner_1_1Jacobi}{gko::preconditioner::Jacobi<>};
    \textcolor{keyword}{using} val\_array = \hyperlink{classgko_1_1Array}{gko::Array<double>};

    std::shared\_ptr<gko::Executor> exec = gko::ReferenceExecutor::create();

    \textcolor{keyword}{auto} b = vec::create(exec, \hyperlink{structgko_1_1dim}{gko::dim<2>}(num\_rows, 1),
                         val\_array::view(exec, num\_rows, rhs.data()), 1);
    \textcolor{keyword}{auto} x = vec::create(exec, \hyperlink{structgko_1_1dim}{gko::dim<2>}(num\_rows, 1));
    \textcolor{keyword}{auto} A = mtx::create(exec, \hyperlink{structgko_1_1dim}{gko::dim<2>}(num\_rows),
                         system\_matrix.n\_nonzero\_elements());
    mtx::value\_type *values = A->get\_values();
    mtx::index\_type *row\_ptr = A->get\_row\_ptrs();
    mtx::index\_type *col\_idx = A->get\_col\_idxs();

    row\_ptr[0] = 0;
    \textcolor{keywordflow}{for} (\textcolor{keyword}{auto} row = 1; row <= num\_rows; ++row) \{
        row\_ptr[row] = row\_ptr[row - 1] + system\_matrix.get\_row\_length(row - 1);
    \}

    std::vector<mtx::index\_type> ptrs(num\_rows + 1);
    std::copy(A->get\_row\_ptrs(), A->get\_row\_ptrs() + num\_rows + 1,
              ptrs.begin());
    \textcolor{keywordflow}{for} (\textcolor{keyword}{auto} row = 0; row < system\_matrix.m(); ++row) \{
        \textcolor{keywordflow}{for} (\textcolor{keyword}{auto} p = system\_matrix.begin(row); p != system\_matrix.end(row);
             ++p) \{
            col\_idx[ptrs[row]] = p->column();
            values[ptrs[row]] = p->value();

            ++ptrs[row];
        \}
    \}

    \textcolor{keyword}{auto} solver\_gen =
        bicgstab::build()
            .with\_criteria(
                gko::stop::Iteration::build().with\_max\_iters(1000).on(exec),
                \hyperlink{classgko_1_1stop_1_1ResidualNormReduction}{gko::stop::ResidualNormReduction<>::build}()
                    .with\_reduction\_factor(1e-12)
                    .on(exec))
            .with\_preconditioner(bj::build().on(exec))
            .on(exec);
    \textcolor{keyword}{auto} solver = solver\_gen->generate(gko::give(A));

    solver->apply(gko::lend(b), gko::lend(x));

    std::copy(x->get\_values(), x->get\_values() + num\_rows, solution.begin());

    \textcolor{comment}{/******************************************************}
\textcolor{comment}{     * deal.ii internal solver. Here for reference.}
\textcolor{comment}{     SolverControl           solver\_control (1000, 1e-12);}
\textcolor{comment}{     SolverBicgstab<>        bicgstab (solver\_control);}
\textcolor{comment}{}
\textcolor{comment}{     PreconditionJacobi<> preconditioner;}
\textcolor{comment}{     preconditioner.initialize(system\_matrix, 1.0);}
\textcolor{comment}{}
\textcolor{comment}{     bicgstab.solve (system\_matrix, solution, system\_rhs,}
\textcolor{comment}{                     preconditioner);}
\textcolor{comment}{    *******************************************************/}

    hanging\_node\_constraints.distribute(solution);
\}


\textcolor{keyword}{template} <\textcolor{keywordtype}{int} dim>
\textcolor{keywordtype}{void} AdvectionProblem<dim>::refine\_grid()
\{
    Vector<float> estimated\_error\_per\_cell(triangulation.n\_active\_cells());

    GradientEstimation::estimate(dof\_handler, solution,
                                 estimated\_error\_per\_cell);

    GridRefinement::refine\_and\_coarsen\_fixed\_number(
        triangulation, estimated\_error\_per\_cell, 0.5, 0.03);

    triangulation.execute\_coarsening\_and\_refinement();
\}


\textcolor{keyword}{template} <\textcolor{keywordtype}{int} dim>
\textcolor{keywordtype}{void} AdvectionProblem<dim>::output\_results(\textcolor{keyword}{const} \textcolor{keywordtype}{unsigned} \textcolor{keywordtype}{int} cycle)\textcolor{keyword}{ const}
\textcolor{keyword}{}\{
    \{
        GridOut grid\_out;
        std::ofstream output(\textcolor{stringliteral}{"grid-"} + std::to\_string(cycle) + \textcolor{stringliteral}{".eps"});
        grid\_out.write\_eps(triangulation, output);
    \}

    \{
        DataOut<dim> data\_out;
        data\_out.attach\_dof\_handler(dof\_handler);
        data\_out.add\_data\_vector(solution, \textcolor{stringliteral}{"solution"});
        data\_out.build\_patches();

        std::ofstream output(\textcolor{stringliteral}{"solution-"} + std::to\_string(cycle) + \textcolor{stringliteral}{".vtk"});
        data\_out.write\_vtk(output);
    \}
\}


\textcolor{keyword}{template} <\textcolor{keywordtype}{int} dim>
\textcolor{keywordtype}{void} AdvectionProblem<dim>::run()
\{
    \textcolor{keywordflow}{for} (\textcolor{keywordtype}{unsigned} \textcolor{keywordtype}{int} cycle = 0; cycle < 6; ++cycle) \{
        std::cout << \textcolor{stringliteral}{"Cycle "} << cycle << \textcolor{charliteral}{':'} << std::endl;

        \textcolor{keywordflow}{if} (cycle == 0) \{
            GridGenerator::hyper\_cube(triangulation, -1, 1);
            triangulation.refine\_global(4);
        \} \textcolor{keywordflow}{else} \{
            refine\_grid();
        \}


        std::cout << \textcolor{stringliteral}{"   Number of active cells:       "}
                  << triangulation.n\_active\_cells() << std::endl;

        setup\_system();

        std::cout << \textcolor{stringliteral}{"   Number of degrees of freedom: "} << dof\_handler.n\_dofs()
                  << std::endl;

        assemble\_system();
        solve();
        output\_results(cycle);
    \}
\}



\textcolor{keyword}{template} <\textcolor{keywordtype}{int} dim>
GradientEstimation::EstimateScratchData<dim>::EstimateScratchData(
    \textcolor{keyword}{const} FiniteElement<dim> &fe, \textcolor{keyword}{const} Vector<double> &solution,
    Vector<float> &error\_per\_cell)
    : fe\_midpoint\_value(fe, QMidpoint<dim>(),
                        update\_values | update\_quadrature\_points),
      solution(solution),
      error\_per\_cell(error\_per\_cell)
\{\}


\textcolor{keyword}{template} <\textcolor{keywordtype}{int} dim>
GradientEstimation::EstimateScratchData<dim>::EstimateScratchData(
    \textcolor{keyword}{const} EstimateScratchData &scratch\_data)
    : fe\_midpoint\_value(scratch\_data.fe\_midpoint\_value.get\_fe(),
                        scratch\_data.fe\_midpoint\_value.get\_quadrature(),
                        update\_values | update\_quadrature\_points),
      solution(scratch\_data.solution),
      error\_per\_cell(scratch\_data.error\_per\_cell)
\{\}


\textcolor{keyword}{template} <\textcolor{keywordtype}{int} dim>
\textcolor{keywordtype}{void} GradientEstimation::estimate(\textcolor{keyword}{const} DoFHandler<dim> &dof\_handler,
                                  \textcolor{keyword}{const} Vector<double> &solution,
                                  Vector<float> &error\_per\_cell)
\{
    Assert(error\_per\_cell.size() ==
               dof\_handler.get\_triangulation().n\_active\_cells(),
           ExcInvalidVectorLength(
               error\_per\_cell.size(),
               dof\_handler.get\_triangulation().n\_active\_cells()));

    WorkStream::run(dof\_handler.begin\_active(), dof\_handler.end(),
                    &GradientEstimation::template estimate\_cell<dim>,
                    std::function<void(const EstimateCopyData &)>(),
                    EstimateScratchData<dim>(dof\_handler.get\_fe(), solution,
                                             error\_per\_cell),
                    EstimateCopyData());
\}


\textcolor{keyword}{template} <\textcolor{keywordtype}{int} dim>
\textcolor{keywordtype}{void} GradientEstimation::estimate\_cell(
    \textcolor{keyword}{const} \textcolor{keyword}{typename} DoFHandler<dim>::active\_cell\_iterator &cell,
    EstimateScratchData<dim> &scratch\_data, \textcolor{keyword}{const} EstimateCopyData &)
\{
    Tensor<2, dim> Y;


    std::vector<typename DoFHandler<dim>::active\_cell\_iterator>
        active\_neighbors;
    active\_neighbors.reserve(GeometryInfo<dim>::faces\_per\_cell *
                             GeometryInfo<dim>::max\_children\_per\_face);

    scratch\_data.fe\_midpoint\_value.reinit(cell);

    Tensor<1, dim> projected\_gradient;


    active\_neighbors.clear();
    \textcolor{keywordflow}{for} (\textcolor{keywordtype}{unsigned} \textcolor{keywordtype}{int} face\_no = 0; face\_no < GeometryInfo<dim>::faces\_per\_cell;
         ++face\_no)
        \textcolor{keywordflow}{if} (!cell->at\_boundary(face\_no)) \{
            \textcolor{keyword}{const} \textcolor{keyword}{typename} DoFHandler<dim>::face\_iterator face =
                cell->face(face\_no);
            \textcolor{keyword}{const} \textcolor{keyword}{typename} DoFHandler<dim>::cell\_iterator neighbor =
                cell->neighbor(face\_no);

            \textcolor{keywordflow}{if} (neighbor->active())
                active\_neighbors.push\_back(neighbor);
            \textcolor{keywordflow}{else} \{
                \textcolor{keywordflow}{if} (dim == 1) \{
                    \textcolor{keyword}{typename} DoFHandler<dim>::cell\_iterator neighbor\_child =
                        neighbor;
                    \textcolor{keywordflow}{while} (neighbor\_child->has\_children())
                        neighbor\_child =
                            neighbor\_child->child(face\_no == 0 ? 1 : 0);

                    Assert(
                        neighbor\_child->neighbor(face\_no == 0 ? 1 : 0) == cell,
                        ExcInternalError());

                    active\_neighbors.push\_back(neighbor\_child);
                \} \textcolor{keywordflow}{else}
                    \textcolor{keywordflow}{for} (\textcolor{keywordtype}{unsigned} \textcolor{keywordtype}{int} subface\_no = 0;
                         subface\_no < face->n\_children(); ++subface\_no)
                        active\_neighbors.push\_back(
                            cell->neighbor\_child\_on\_subface(face\_no,
                                                            subface\_no));
            \}
        \}

    \textcolor{keyword}{const} Point<dim> this\_center =
        scratch\_data.fe\_midpoint\_value.quadrature\_point(0);

    std::vector<double> this\_midpoint\_value(1);
    scratch\_data.fe\_midpoint\_value.get\_function\_values(scratch\_data.solution,
                                                       this\_midpoint\_value);


    std::vector<double> neighbor\_midpoint\_value(1);
    \textcolor{keyword}{typename} std::vector<typename DoFHandler<dim>::active\_cell\_iterator>::
        const\_iterator neighbor\_ptr = active\_neighbors.begin();
    \textcolor{keywordflow}{for} (; neighbor\_ptr != active\_neighbors.end(); ++neighbor\_ptr) \{
        \textcolor{keyword}{const} \textcolor{keyword}{typename} DoFHandler<dim>::active\_cell\_iterator neighbor =
            *neighbor\_ptr;

        scratch\_data.fe\_midpoint\_value.reinit(neighbor);
        \textcolor{keyword}{const} Point<dim> neighbor\_center =
            scratch\_data.fe\_midpoint\_value.quadrature\_point(0);

        scratch\_data.fe\_midpoint\_value.get\_function\_values(
            scratch\_data.solution, neighbor\_midpoint\_value);

        Tensor<1, dim> y = neighbor\_center - this\_center;
        \textcolor{keyword}{const} \textcolor{keywordtype}{double} distance = y.norm();
        y /= distance;

        \textcolor{keywordflow}{for} (\textcolor{keywordtype}{unsigned} \textcolor{keywordtype}{int} i = 0; i < dim; ++i)
            \textcolor{keywordflow}{for} (\textcolor{keywordtype}{unsigned} \textcolor{keywordtype}{int} j = 0; j < dim; ++j) Y[i][j] += y[i] * y[j];

        projected\_gradient +=
            (neighbor\_midpoint\_value[0] - this\_midpoint\_value[0]) / distance *
            y;
    \}

    AssertThrow(determinant(Y) != 0, ExcInsufficientDirections());

    \textcolor{keyword}{const} Tensor<2, dim> Y\_inverse = invert(Y);

    Tensor<1, dim> gradient = Y\_inverse * projected\_gradient;

    scratch\_data.error\_per\_cell(cell->active\_cell\_index()) =
        (std::pow(cell->diameter(), 1 + 1.0 * dim / 2) *
         std::sqrt(gradient.norm\_square()));
\}
\}  \textcolor{comment}{// namespace Step9}



\textcolor{keywordtype}{int} main()
\{
    \textcolor{keywordflow}{try} \{
        dealii::MultithreadInfo::set\_thread\_limit();

        Step9::AdvectionProblem<2> advection\_problem\_2d;
        advection\_problem\_2d.run();
    \} \textcolor{keywordflow}{catch} (std::exception &exc) \{
        std::cerr << std::endl
                  << std::endl
                  << \textcolor{stringliteral}{"----------------------------------------------------"}
                  << std::endl;
        std::cerr << \textcolor{stringliteral}{"Exception on processing: "} << std::endl
                  << exc.what() << std::endl
                  << \textcolor{stringliteral}{"Aborting!"} << std::endl
                  << \textcolor{stringliteral}{"----------------------------------------------------"}
                  << std::endl;
        \textcolor{keywordflow}{return} 1;
    \} \textcolor{keywordflow}{catch} (...) \{
        std::cerr << std::endl
                  << std::endl
                  << \textcolor{stringliteral}{"----------------------------------------------------"}
                  << std::endl;
        std::cerr << \textcolor{stringliteral}{"Unknown exception!"} << std::endl
                  << \textcolor{stringliteral}{"Aborting!"} << std::endl
                  << \textcolor{stringliteral}{"----------------------------------------------------"}
                  << std::endl;
        \textcolor{keywordflow}{return} 1;
    \}

    \textcolor{keywordflow}{return} 0;
\}
\end{DoxyCodeInclude}
 