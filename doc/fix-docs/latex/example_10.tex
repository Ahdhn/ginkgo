The papi logging example.

This example depends on \hyperlink{example_9}{example-\/9}.

 \label{_Intro}%
 \label{_Introduction}%
\section*{Introduction}

\label{_Abouttheexample}%
\subsubsection*{About the example }

\label{_CommProg}%
 \section*{The commented program}


\begin{DoxyCode}
\textcolor{preprocessor}{#include <include/ginkgo/ginkgo.hpp>}


\textcolor{preprocessor}{#include <papi.h>}
\textcolor{preprocessor}{#include <fstream>}
\textcolor{preprocessor}{#include <iostream>}
\textcolor{preprocessor}{#include <string>}
\textcolor{preprocessor}{#include <thread>}


\textcolor{keyword}{namespace }\{


\textcolor{keywordtype}{void} papi\_add\_event(\textcolor{keyword}{const} std::string &event\_name, \textcolor{keywordtype}{int} &eventset)
\{
    \textcolor{keywordtype}{int} code;
    \textcolor{keywordtype}{int} ret\_val = PAPI\_event\_name\_to\_code(event\_name.c\_str(), &code);
    \textcolor{keywordflow}{if} (PAPI\_OK != ret\_val) \{
        std::cerr << \textcolor{stringliteral}{"Error at PAPI\_name\_to\_code()"} << std::endl;
        std::exit(-1);
    \}

    ret\_val = PAPI\_add\_event(eventset, code);
    \textcolor{keywordflow}{if} (PAPI\_OK != ret\_val) \{
        std::cerr << \textcolor{stringliteral}{"Error at PAPI\_name\_to\_code()"} << std::endl;
        std::exit(-1);
    \}
\}


\textcolor{keyword}{template} <\textcolor{keyword}{typename} T>
std::string to\_string(T *ptr)
\{
    std::ostringstream os;
    os << reinterpret\_cast<gko::uintptr>(ptr);
    \textcolor{keywordflow}{return} os.str();
\}


\}  \textcolor{comment}{// namespace}


\textcolor{keywordtype}{int} init\_papi\_counters(std::string solver\_name, std::string A\_name)
\{
\end{DoxyCode}


Initialize P\+A\+PI, add events and start it up


\begin{DoxyCode}
    \textcolor{keywordtype}{int} eventset = PAPI\_NULL;
    \textcolor{keywordtype}{int} ret\_val = PAPI\_library\_init(PAPI\_VER\_CURRENT);
    \textcolor{keywordflow}{if} (ret\_val != PAPI\_VER\_CURRENT) \{
        std::cerr << \textcolor{stringliteral}{"Error at PAPI\_library\_init()"} << std::endl;
        std::exit(-1);
    \}
    ret\_val = PAPI\_create\_eventset(&eventset);
    \textcolor{keywordflow}{if} (PAPI\_OK != ret\_val) \{
        std::cerr << \textcolor{stringliteral}{"Error at PAPI\_create\_eventset()"} << std::endl;
        std::exit(-1);
    \}

    std::string simple\_apply\_string(\textcolor{stringliteral}{"sde:::ginkgo0::linop\_apply\_completed::"});
    std::string advanced\_apply\_string(
        \textcolor{stringliteral}{"sde:::ginkgo0::linop\_advanced\_apply\_completed::"});
    papi\_add\_event(simple\_apply\_string + solver\_name, eventset);
    papi\_add\_event(simple\_apply\_string + A\_name, eventset);
    papi\_add\_event(advanced\_apply\_string + A\_name, eventset);

    ret\_val = PAPI\_start(eventset);
    \textcolor{keywordflow}{if} (PAPI\_OK != ret\_val) \{
        std::cerr << \textcolor{stringliteral}{"Error at PAPI\_start()"} << std::endl;
        std::exit(-1);
    \}
    \textcolor{keywordflow}{return} eventset;
\}


\textcolor{keywordtype}{void} print\_papi\_counters(\textcolor{keywordtype}{int} eventset)
\{
\end{DoxyCode}


Stop P\+A\+PI and read the linop\+\_\+apply\+\_\+completed event for all of them


\begin{DoxyCode}
\textcolor{keywordtype}{long} \textcolor{keywordtype}{long} \textcolor{keywordtype}{int} values[3];
\textcolor{keywordtype}{int} ret\_val = PAPI\_stop(eventset, values);
\textcolor{keywordflow}{if} (PAPI\_OK != ret\_val) \{
    std::cerr << \textcolor{stringliteral}{"Error at PAPI\_stop()"} << std::endl;
    std::exit(-1);
\}

PAPI\_shutdown();
\end{DoxyCode}


Print all values returned from P\+A\+PI


\begin{DoxyCode}
    std::cout << \textcolor{stringliteral}{"PAPI SDE counters:"} << std::endl;
    std::cout << \textcolor{stringliteral}{"solver did "} << values[0] << \textcolor{stringliteral}{" applies."} << std::endl;
    std::cout << \textcolor{stringliteral}{"A did "} << values[1] << \textcolor{stringliteral}{" simple applies."} << std::endl;
    std::cout << \textcolor{stringliteral}{"A did "} << values[2] << \textcolor{stringliteral}{" advanced applies."} << std::endl;
\}


\textcolor{keywordtype}{int} main(\textcolor{keywordtype}{int} argc, \textcolor{keywordtype}{char} *argv[])
\{
\end{DoxyCode}


Some shortcuts


\begin{DoxyCode}
\textcolor{keyword}{using} vec = \hyperlink{classgko_1_1matrix_1_1Dense}{gko::matrix::Dense<>};
\textcolor{keyword}{using} mtx = \hyperlink{classgko_1_1matrix_1_1Csr}{gko::matrix::Csr<>};
\textcolor{keyword}{using} cg = \hyperlink{classgko_1_1solver_1_1Cg}{gko::solver::Cg<>};
\end{DoxyCode}


Print version information


\begin{DoxyCode}
std::cout << \hyperlink{classgko_1_1version__info_a6daeb8a087cfb57fa055526fc133d8eb}{gko::version\_info::get}() << std::endl;
\end{DoxyCode}


Figure out where to run the code


\begin{DoxyCode}
std::shared\_ptr<gko::Executor> exec;
\textcolor{keywordflow}{if} (argc == 1 || std::string(argv[1]) == \textcolor{stringliteral}{"reference"}) \{
    exec = gko::ReferenceExecutor::create();
\} \textcolor{keywordflow}{else} \textcolor{keywordflow}{if} (argc == 2 && std::string(argv[1]) == \textcolor{stringliteral}{"omp"}) \{
    exec = \hyperlink{classgko_1_1OmpExecutor_a33ca05fdd0fc928ee262fc9425304874}{gko::OmpExecutor::create}();
\} \textcolor{keywordflow}{else} \textcolor{keywordflow}{if} (argc == 2 && std::string(argv[1]) == \textcolor{stringliteral}{"cuda"} &&
           \hyperlink{classgko_1_1CudaExecutor_aef0258494d14de0e56149b920c5173e5}{gko::CudaExecutor::get\_num\_devices}() > 0) \{
    exec = \hyperlink{classgko_1_1CudaExecutor_a2718a92034350650ef406ffdb60db090}{gko::CudaExecutor::create}(0, 
      \hyperlink{classgko_1_1OmpExecutor_a33ca05fdd0fc928ee262fc9425304874}{gko::OmpExecutor::create}());
\} \textcolor{keywordflow}{else} \{
    std::cerr << \textcolor{stringliteral}{"Usage: "} << argv[0] << \textcolor{stringliteral}{" [executor]"} << std::endl;
    std::exit(-1);
\}
\end{DoxyCode}


Read data


\begin{DoxyCode}
\textcolor{keyword}{auto} A = \hyperlink{namespacegko_a3ce296f73db0ff398bdea6009a3a5c58}{share}(gko::read<mtx>(std::ifstream(\textcolor{stringliteral}{"data/A.mtx"}), exec));
\textcolor{keyword}{auto} b = gko::read<vec>(std::ifstream(\textcolor{stringliteral}{"data/b.mtx"}), exec);
\textcolor{keyword}{auto} x = gko::read<vec>(std::ifstream(\textcolor{stringliteral}{"data/x0.mtx"}), exec);
\end{DoxyCode}


Generate solver


\begin{DoxyCode}
\textcolor{keyword}{auto} solver\_gen =
    cg::build()
        .with\_criteria(
            gko::stop::Iteration::build().with\_max\_iters(20u).on(exec),
            \hyperlink{classgko_1_1stop_1_1ResidualNormReduction}{gko::stop::ResidualNormReduction<>::build}()
                .with\_reduction\_factor(1e-20)
                .on(exec))
        .on(exec);
\textcolor{keyword}{auto} solver = solver\_gen->generate(A);
\end{DoxyCode}


In this example, we split as much as possible the Ginkgo solver/logger and the P\+A\+PI interface. Note that the P\+A\+PI ginkgo namespaces are of the form sde\+:\+::ginkgo$<$x$>$ where $<$x$>$ starts from 0 and is incremented with every new P\+A\+PI logger.


\begin{DoxyCode}
\textcolor{keywordtype}{int} eventset =
    init\_papi\_counters(to\_string(solver.get()), to\_string(A.get()));
\end{DoxyCode}


Create a P\+A\+PI logger and add it to relevant Lin\+Ops


\begin{DoxyCode}
\textcolor{keyword}{auto} logger = gko::log::Papi<>::create(
    exec, gko::log::Logger::linop\_apply\_completed\_mask |
              gko::log::Logger::linop\_advanced\_apply\_completed\_mask);
solver->add\_logger(logger);
A->add\_logger(logger);
\end{DoxyCode}


Solve system


\begin{DoxyCode}
solver->apply(\hyperlink{namespacegko_aa8cb4876b72e5e1036ea9575443c439b}{lend}(b), \hyperlink{namespacegko_aa8cb4876b72e5e1036ea9575443c439b}{lend}(x));
\end{DoxyCode}


Stop P\+A\+PI event gathering and print the counters


\begin{DoxyCode}
print\_papi\_counters(eventset);
\end{DoxyCode}


Print solution


\begin{DoxyCode}
std::cout << \textcolor{stringliteral}{"Solution (x): \(\backslash\)n"};
\hyperlink{namespacegko_a859dc47a462721d83728d91ab7fa2148}{write}(std::cout, \hyperlink{namespacegko_aa8cb4876b72e5e1036ea9575443c439b}{lend}(x));
\end{DoxyCode}


Calculate residual


\begin{DoxyCode}
    \textcolor{keyword}{auto} \hyperlink{namespacegko_a0059e27f8f4bc348ff65c1e60caf47c8}{one} = gko::initialize<vec>(\{1.0\}, exec);
    \textcolor{keyword}{auto} neg\_one = gko::initialize<vec>(\{-1.0\}, exec);
    \textcolor{keyword}{auto} res = gko::initialize<vec>(\{0.0\}, exec);
    A->apply(\hyperlink{namespacegko_aa8cb4876b72e5e1036ea9575443c439b}{lend}(one), \hyperlink{namespacegko_aa8cb4876b72e5e1036ea9575443c439b}{lend}(x), \hyperlink{namespacegko_aa8cb4876b72e5e1036ea9575443c439b}{lend}(neg\_one), \hyperlink{namespacegko_aa8cb4876b72e5e1036ea9575443c439b}{lend}(b));
    b->compute\_norm2(\hyperlink{namespacegko_aa8cb4876b72e5e1036ea9575443c439b}{lend}(res));

    std::cout << \textcolor{stringliteral}{"Residual norm sqrt(r^T r): \(\backslash\)n"};
    \hyperlink{namespacegko_a859dc47a462721d83728d91ab7fa2148}{write}(std::cout, \hyperlink{namespacegko_aa8cb4876b72e5e1036ea9575443c439b}{lend}(res));
\}
\end{DoxyCode}
 \label{_Results}%
\section*{Results}

\label{_Commentsaboutprogramminganddebugging}%
\paragraph*{Comments about programming and debugging }

\label{_PlainProg}%
 \section*{The plain program}


\begin{DoxyCodeInclude}
\end{DoxyCodeInclude}
 