The custom matrix format example.

This example depends on simple-\/solver, poisson-\/solver, three-\/pt-\/stencil-\/solver, .

 \label{_Intro}%
 \label{_Introduction}%
\section*{Introduction}

This example solves a 1D Poisson equation\+:

$ u : [0, 1] \rightarrow R\\ u'' = f\\ u(0) = u0\\ u(1) = u1 $

using a finite difference method on an equidistant grid with {\ttfamily K} discretization points ({\ttfamily K} can be controlled with a command line parameter). The discretization is done via the second order Taylor polynomial\+:

$ u(x + h) = u(x) - u'(x)h + 1/2 u''(x)h^2 + O(h^3)\\ u(x - h) = u(x) + u'(x)h + 1/2 u''(x)h^2 + O(h^3) / +\\ ---------------------- \\ -u(x - h) + 2u(x) + -u(x + h) = -f(x)h^2 + O(h^3) $

For an equidistant grid with K \char`\"{}inner\char`\"{} discretization points $x1, ..., xk, $and step size $ h = 1 / (K + 1)$, the formula produces a system of linear equations

$ 2u_1 - u_2 = -f_1 h^2 + u0\\ -u_(k-1) + 2u_k - u_(k+1) = -f_k h^2, k = 2, ..., K - 1\\ -u_(K-1) + 2u_K = -f_K h^2 + u1\\ $

which is then solved using Ginkgo\textquotesingle{}s implementation of the CG method preconditioned with block-\/\+Jacobi. It is also possible to specify on which executor Ginkgo will solve the system via the command line. The function $`f` $is set to $`f(x) = 6x`$ (making the solution $`u(x) = x^3`$), but that can be changed in the {\ttfamily main} function.

The intention of this example is to show how a custom linear operator can be created and integrated into Ginkgo to achieve performance benefits.

\label{_Abouttheexample}%
\subsubsection*{About the example }

\label{_CommProg}%
 \section*{The commented program}


\begin{DoxyCode}
\textcolor{preprocessor}{#include <iostream>}
\textcolor{preprocessor}{#include <map>}
\textcolor{preprocessor}{#include <string>}


\textcolor{preprocessor}{#include <omp.h>}
\textcolor{preprocessor}{#include <ginkgo/ginkgo.hpp>}
\end{DoxyCode}


A C\+U\+DA kernel implementing the stencil, which will be used if running on the C\+U\+DA executor. Unfortunately, N\+V\+CC has serious problems interpreting some parts of Ginkgo\textquotesingle{}s code, so the kernel has to be compiled separately.


\begin{DoxyCode}
\textcolor{keyword}{extern} \textcolor{keywordtype}{void} stencil\_kernel(std::size\_t size, \textcolor{keyword}{const} \textcolor{keywordtype}{double} *coefs,
                           \textcolor{keyword}{const} \textcolor{keywordtype}{double} *b, \textcolor{keywordtype}{double} *x);
\end{DoxyCode}


A stencil matrix class representing the 3pt stencil linear operator. We include the \hyperlink{classgko_1_1EnableLinOp}{gko\+::\+Enable\+Lin\+Op} mixin which implements the entire Lin\+Op interface, except the two apply\+\_\+impl methods, which get called inside the default implementation of apply (after argument verification) to perform the actual application of the linear operator. In addition, it includes the implementation of the entire Polymorphic\+Object interface.

It also includes the \hyperlink{classgko_1_1EnableCreateMethod}{gko\+::\+Enable\+Create\+Method} mixin which provides a default implementation of the static create method. This method will forward all its arguments to the constructor to create the object, and return an std\+::unique\+\_\+ptr to the created object.


\begin{DoxyCode}
\textcolor{keyword}{class }StencilMatrix : \textcolor{keyword}{public} \hyperlink{classgko_1_1EnableLinOp}{gko::EnableLinOp}<StencilMatrix>,
                      \textcolor{keyword}{public} \hyperlink{classgko_1_1EnableCreateMethod}{gko::EnableCreateMethod}<StencilMatrix> \{
\textcolor{keyword}{public}:
\end{DoxyCode}


This constructor will be called by the create method. Here we initialize the coefficients of the stencil.


\begin{DoxyCode}
    StencilMatrix(std::shared\_ptr<const gko::Executor> exec,
                  \hyperlink{namespacegko_a6e5c95df0ae4e47aab2f604a22d98ee7}{gko::size\_type} size = 0, \textcolor{keywordtype}{double} left = -1.0,
                  \textcolor{keywordtype}{double} center = 2.0, \textcolor{keywordtype}{double} right = -1.0)
        : \hyperlink{namespacegko}{gko}::EnableLinOp<StencilMatrix>(exec, \hyperlink{namespacegko}{gko}::dim<2>\{size\}),
          coefficients(exec, \{left, center, right\})
    \{\}

\textcolor{keyword}{protected}:
    \textcolor{keyword}{using} vec = \hyperlink{classgko_1_1matrix_1_1Dense}{gko::matrix::Dense<>};
    \textcolor{keyword}{using} coef\_type = \hyperlink{classgko_1_1Array}{gko::Array<double>};
\end{DoxyCode}


Here we implement the application of the linear operator, x = A $\ast$ b. apply\+\_\+impl will be called by the apply method, after the arguments have been moved to the correct executor and the operators checked for conforming sizes.

For simplicity, we assume that there is always only one right hand side and the stride of consecutive elements in the vectors is 1 (both of these are always true in this example).


\begin{DoxyCode}
\textcolor{keywordtype}{void} apply\_impl(\textcolor{keyword}{const} \hyperlink{classgko_1_1LinOp}{gko::LinOp} *b, \hyperlink{classgko_1_1LinOp}{gko::LinOp} *x)\textcolor{keyword}{ const override}
\textcolor{keyword}{}\{
\end{DoxyCode}


we only implement the operator for dense R\+HS. gko\+::as will throw an exception if its argument is not Dense.


\begin{DoxyCode}
\textcolor{keyword}{auto} dense\_b = gko::as<vec>(b);
\textcolor{keyword}{auto} dense\_x = gko::as<vec>(x);
\end{DoxyCode}


we need separate implementations depending on the executor, so we create an operation which maps the call to the correct implementation


\begin{DoxyCode}
\textcolor{keyword}{struct }stencil\_operation : \hyperlink{classgko_1_1Operation}{gko::Operation} \{
    stencil\_operation(\textcolor{keyword}{const} coef\_type &coefficients, \textcolor{keyword}{const} vec *b,
                      vec *x)
        : coefficients\{coefficients\}, b\{b\}, x\{x\}
    \{\}
\end{DoxyCode}


Open\+MP implementation


\begin{DoxyCode}
            \textcolor{keywordtype}{void} run(std::shared\_ptr<const gko::OmpExecutor>)\textcolor{keyword}{ const override}
\textcolor{keyword}{            }\{
                \textcolor{keyword}{auto} b\_values = b->get\_const\_values();
                \textcolor{keyword}{auto} x\_values = x->get\_values();
\textcolor{preprocessor}{#pragma omp parallel for}
                \textcolor{keywordflow}{for} (std::size\_t i = 0; i < x->\hyperlink{classgko_1_1LinOp_a31b3c003388eb0b95393154f68c2b98d}{get\_size}()[0]; ++i) \{
                    \textcolor{keyword}{auto} coefs = coefficients.get\_const\_data();
                    \textcolor{keyword}{auto} result = coefs[1] * b\_values[i];
                    \textcolor{keywordflow}{if} (i > 0) \{
                        result += coefs[0] * b\_values[i - 1];
                    \}
                    \textcolor{keywordflow}{if} (i < x->get\_size()[0] - 1) \{
                        result += coefs[2] * b\_values[i + 1];
                    \}
                    x\_values[i] = result;
                \}
            \}
\end{DoxyCode}


C\+U\+DA implementation


\begin{DoxyCode}
\textcolor{keywordtype}{void} run(std::shared\_ptr<const gko::CudaExecutor>)\textcolor{keyword}{ const override}
\textcolor{keyword}{}\{
    stencil\_kernel(x->\hyperlink{classgko_1_1LinOp_a31b3c003388eb0b95393154f68c2b98d}{get\_size}()[0], coefficients.get\_const\_data(),
                   b->get\_const\_values(), x->get\_values());
\}
\end{DoxyCode}


We do not provide an implementation for reference executor. If not provided, Ginkgo will use the implementation for the Open\+MP executor when calling it in the reference executor.


\begin{DoxyCode}
        \textcolor{keyword}{const} coef\_type &coefficients;
        \textcolor{keyword}{const} vec *b;
        vec *x;
    \};
    this->get\_executor()->run(
        stencil\_operation(coefficients, dense\_b, dense\_x));
\}
\end{DoxyCode}


There is also a version of the apply function which does the operation x = alpha $\ast$ A $\ast$ b + beta $\ast$ x. This function is commonly used and can often be better optimized than implementing it using x = A $\ast$ b. However, for simplicity, we will implement it exactly like that in this example.


\begin{DoxyCode}
    \textcolor{keywordtype}{void} apply\_impl(\textcolor{keyword}{const} \hyperlink{classgko_1_1LinOp}{gko::LinOp} *alpha, \textcolor{keyword}{const} \hyperlink{classgko_1_1LinOp}{gko::LinOp} *b,
                    \textcolor{keyword}{const} \hyperlink{classgko_1_1LinOp}{gko::LinOp} *beta, \hyperlink{classgko_1_1LinOp}{gko::LinOp} *x)\textcolor{keyword}{ const override}
\textcolor{keyword}{    }\{
        \textcolor{keyword}{auto} dense\_b = gko::as<vec>(b);
        \textcolor{keyword}{auto} dense\_x = gko::as<vec>(x);
        \textcolor{keyword}{auto} tmp\_x = dense\_x->clone();
        this->apply\_impl(b, lend(tmp\_x));
        dense\_x->scale(beta);
        dense\_x->add\_scaled(alpha, lend(tmp\_x));
    \}

\textcolor{keyword}{private}:
    coef\_type coefficients;
\};
\end{DoxyCode}


Creates a stencil matrix in C\+SR format for the given number of discretization points.


\begin{DoxyCode}
\textcolor{keywordtype}{void} generate\_stencil\_matrix(\hyperlink{classgko_1_1matrix_1_1Csr}{gko::matrix::Csr<>} *matrix)
\{
    \textcolor{keyword}{const} \textcolor{keyword}{auto} discretization\_points = matrix->\hyperlink{classgko_1_1LinOp_a31b3c003388eb0b95393154f68c2b98d}{get\_size}()[0];
    \textcolor{keyword}{auto} row\_ptrs = matrix->\hyperlink{classgko_1_1matrix_1_1Csr_a068e5158cf282fa977f0a137f8cd7f03}{get\_row\_ptrs}();
    \textcolor{keyword}{auto} col\_idxs = matrix->\hyperlink{classgko_1_1matrix_1_1Csr_a81c6294177a1be4873804c8a85a9fc64}{get\_col\_idxs}();
    \textcolor{keyword}{auto} values = matrix->\hyperlink{classgko_1_1matrix_1_1Csr_a929b0a194e6aeb1252b8e6781d162e83}{get\_values}();
    \textcolor{keywordtype}{int} pos = 0;
    \textcolor{keyword}{const} \textcolor{keywordtype}{double} coefs[] = \{-1, 2, -1\};
    row\_ptrs[0] = pos;
    \textcolor{keywordflow}{for} (\textcolor{keywordtype}{int} i = 0; i < discretization\_points; ++i) \{
        \textcolor{keywordflow}{for} (\textcolor{keyword}{auto} ofs : \{-1, 0, 1\}) \{
            \textcolor{keywordflow}{if} (0 <= i + ofs && i + ofs < discretization\_points) \{
                values[pos] = coefs[ofs + 1];
                col\_idxs[pos] = i + ofs;
                ++pos;
            \}
        \}
        row\_ptrs[i + 1] = pos;
    \}
\}
\end{DoxyCode}


Generates the R\+HS vector given {\ttfamily f} and the boundary conditions.


\begin{DoxyCode}
\textcolor{keyword}{template} <\textcolor{keyword}{typename} Closure>
\textcolor{keywordtype}{void} generate\_rhs(Closure f, \textcolor{keywordtype}{double} u0, \textcolor{keywordtype}{double} u1, \hyperlink{classgko_1_1matrix_1_1Dense}{gko::matrix::Dense<>} *rhs)
\{
    \textcolor{keyword}{const} \textcolor{keyword}{auto} discretization\_points = rhs->\hyperlink{classgko_1_1LinOp_a31b3c003388eb0b95393154f68c2b98d}{get\_size}()[0];
    \textcolor{keyword}{auto} values = rhs->\hyperlink{classgko_1_1matrix_1_1Dense_a3bc458e02fab8e4c9f60f70bd4d5a4f9}{get\_values}();
    \textcolor{keyword}{const} \textcolor{keyword}{auto} h = 1.0 / (discretization\_points + 1);
    \textcolor{keywordflow}{for} (\textcolor{keywordtype}{int} i = 0; i < discretization\_points; ++i) \{
        \textcolor{keyword}{const} \textcolor{keyword}{auto} xi = (i + 1) * h;
        values[i] = -f(xi) * h * h;
    \}
    values[0] += u0;
    values[discretization\_points - 1] += u1;
\}
\end{DoxyCode}


Prints the solution {\ttfamily u}.


\begin{DoxyCode}
\textcolor{keywordtype}{void} print\_solution(\textcolor{keywordtype}{double} u0, \textcolor{keywordtype}{double} u1, \textcolor{keyword}{const} \hyperlink{classgko_1_1matrix_1_1Dense}{gko::matrix::Dense<>} *u)
\{
    std::cout << u0 << \textcolor{charliteral}{'\(\backslash\)n'};
    \textcolor{keywordflow}{for} (\textcolor{keywordtype}{int} i = 0; i < u->\hyperlink{classgko_1_1LinOp_a31b3c003388eb0b95393154f68c2b98d}{get\_size}()[0]; ++i) \{
        std::cout << u->\hyperlink{classgko_1_1matrix_1_1Dense_ab83c739c1b11abaecc3bfd89506f6c9c}{get\_const\_values}()[i] << \textcolor{charliteral}{'\(\backslash\)n'};
    \}
    std::cout << u1 << std::endl;
\}
\end{DoxyCode}


Computes the 1-\/norm of the error given the computed {\ttfamily u} and the correct solution function {\ttfamily correct\+\_\+u}.


\begin{DoxyCode}
\textcolor{keyword}{template} <\textcolor{keyword}{typename} Closure>
\textcolor{keywordtype}{double} calculate\_error(\textcolor{keywordtype}{int} discretization\_points, \textcolor{keyword}{const} \hyperlink{classgko_1_1matrix_1_1Dense}{gko::matrix::Dense<>} *u,
                       Closure correct\_u)
\{
    \textcolor{keyword}{const} \textcolor{keyword}{auto} h = 1.0 / (discretization\_points + 1);
    \textcolor{keyword}{auto} error = 0.0;
    \textcolor{keywordflow}{for} (\textcolor{keywordtype}{int} i = 0; i < discretization\_points; ++i) \{
        \textcolor{keyword}{using} std::abs;
        \textcolor{keyword}{const} \textcolor{keyword}{auto} xi = (i + 1) * h;
        error +=
            \hyperlink{namespacegko_a57797fc0a00fd4b7ff34ca4bfc84bc51}{abs}(u->\hyperlink{classgko_1_1matrix_1_1Dense_ab83c739c1b11abaecc3bfd89506f6c9c}{get\_const\_values}()[i] - correct\_u(xi)) / 
      \hyperlink{namespacegko_a57797fc0a00fd4b7ff34ca4bfc84bc51}{abs}(correct\_u(xi));
    \}
    \textcolor{keywordflow}{return} error;
\}


\textcolor{keywordtype}{int} main(\textcolor{keywordtype}{int} argc, \textcolor{keywordtype}{char} *argv[])
\{
\end{DoxyCode}


Some shortcuts


\begin{DoxyCode}
\textcolor{keyword}{using} vec = \hyperlink{classgko_1_1matrix_1_1Dense}{gko::matrix::Dense<double>};
\textcolor{keyword}{using} mtx = \hyperlink{classgko_1_1matrix_1_1Csr}{gko::matrix::Csr<double, int>};
\textcolor{keyword}{using} cg = \hyperlink{classgko_1_1solver_1_1Cg}{gko::solver::Cg<double>};

\textcolor{keywordflow}{if} (argc < 2) \{
    std::cerr << \textcolor{stringliteral}{"Usage: "} << argv[0] << \textcolor{stringliteral}{" DISCRETIZATION\_POINTS [executor]"}
              << std::endl;
    std::exit(-1);
\}
\end{DoxyCode}


Get number of discretization points


\begin{DoxyCode}
\textcolor{keyword}{const} \textcolor{keywordtype}{unsigned} \textcolor{keywordtype}{int} discretization\_points =
    argc >= 2 ? std::atoi(argv[1]) : 100u;
\textcolor{keyword}{const} \textcolor{keyword}{auto} executor\_string = argc >= 3 ? argv[2] : \textcolor{stringliteral}{"reference"};
\end{DoxyCode}


Figure out where to run the code


\begin{DoxyCode}
\textcolor{keyword}{const} \textcolor{keyword}{auto} omp = \hyperlink{classgko_1_1OmpExecutor_a33ca05fdd0fc928ee262fc9425304874}{gko::OmpExecutor::create}();
std::map<std::string, std::shared\_ptr<gko::Executor>> exec\_map\{
    \{\textcolor{stringliteral}{"omp"}, omp\},
    \{\textcolor{stringliteral}{"cuda"}, \hyperlink{classgko_1_1CudaExecutor_a2718a92034350650ef406ffdb60db090}{gko::CudaExecutor::create}(0, omp)\},
    \{\textcolor{stringliteral}{"reference"}, gko::ReferenceExecutor::create()\}\};
\end{DoxyCode}


executor where Ginkgo will perform the computation


\begin{DoxyCode}
\textcolor{keyword}{const} \textcolor{keyword}{auto} exec = exec\_map.at(executor\_string);  \textcolor{comment}{// throws if not valid}
\end{DoxyCode}


executor used by the application


\begin{DoxyCode}
\textcolor{keyword}{const} \textcolor{keyword}{auto} app\_exec = exec\_map[\textcolor{stringliteral}{"omp"}];
\end{DoxyCode}


problem\+:


\begin{DoxyCode}
\textcolor{keyword}{auto} correct\_u = [](\textcolor{keywordtype}{double} x) \{ \textcolor{keywordflow}{return} x * x * x; \};
\textcolor{keyword}{auto} f = [](\textcolor{keywordtype}{double} x) \{ \textcolor{keywordflow}{return} 6 * x; \};
\textcolor{keyword}{auto} u0 = correct\_u(0);
\textcolor{keyword}{auto} u1 = correct\_u(1);
\end{DoxyCode}


initialize vectors


\begin{DoxyCode}
\textcolor{keyword}{auto} rhs = vec::create(app\_exec, \hyperlink{structgko_1_1dim}{gko::dim<2>}(discretization\_points, 1));
generate\_rhs(f, u0, u1, lend(rhs));
\textcolor{keyword}{auto} u = vec::create(app\_exec, \hyperlink{structgko_1_1dim}{gko::dim<2>}(discretization\_points, 1));
\textcolor{keywordflow}{for} (\textcolor{keywordtype}{int} i = 0; i < u->\hyperlink{classgko_1_1LinOp_a31b3c003388eb0b95393154f68c2b98d}{get\_size}()[0]; ++i) \{
    u->\hyperlink{classgko_1_1matrix_1_1Dense_a3bc458e02fab8e4c9f60f70bd4d5a4f9}{get\_values}()[i] = 0.0;
\}
\end{DoxyCode}


Generate solver and solve the system


\begin{DoxyCode}
cg::build()
    .with\_criteria(gko::stop::Iteration::build()
                       .with\_max\_iters(discretization\_points)
                       .on(exec),
                   \hyperlink{classgko_1_1stop_1_1ResidualNormReduction}{gko::stop::ResidualNormReduction<>::build}()
                       .with\_reduction\_factor(1e-6)
                       .on(exec))
    .on(exec)
\end{DoxyCode}


notice how our custom Stencil\+Matrix can be used in the same way as any built-\/in type


\begin{DoxyCode}
        ->generate(
            StencilMatrix::create(exec, discretization\_points, -1, 2, -1))
        ->apply(lend(rhs), lend(u));

    print\_solution(u0, u1, lend(u));
    std::cout << \textcolor{stringliteral}{"The average relative error is "}
              << calculate\_error(discretization\_points, lend(u), correct\_u) /
                     discretization\_points
              << std::endl;
\}
\end{DoxyCode}
 \label{_Results}%
\section*{Results}

\label{_Commentsaboutprogramminganddebugging}%
\paragraph*{Comments about programming and debugging }

\label{_PlainProg}%
 \section*{The plain program}


\begin{DoxyCodeInclude}
\textcolor{comment}{/*******************************<GINKGO LICENSE>******************************}
\textcolor{comment}{Copyright (c) 2017-2019, the Ginkgo authors}
\textcolor{comment}{All rights reserved.}
\textcolor{comment}{}
\textcolor{comment}{Redistribution and use in source and binary forms, with or without}
\textcolor{comment}{modification, are permitted provided that the following conditions}
\textcolor{comment}{are met:}
\textcolor{comment}{}
\textcolor{comment}{1. Redistributions of source code must retain the above copyright}
\textcolor{comment}{notice, this list of conditions and the following disclaimer.}
\textcolor{comment}{}
\textcolor{comment}{2. Redistributions in binary form must reproduce the above copyright}
\textcolor{comment}{notice, this list of conditions and the following disclaimer in the}
\textcolor{comment}{documentation and/or other materials provided with the distribution.}
\textcolor{comment}{}
\textcolor{comment}{3. Neither the name of the copyright holder nor the names of its}
\textcolor{comment}{contributors may be used to endorse or promote products derived from}
\textcolor{comment}{this software without specific prior written permission.}
\textcolor{comment}{}
\textcolor{comment}{THIS SOFTWARE IS PROVIDED BY THE COPYRIGHT HOLDERS AND CONTRIBUTORS "AS}
\textcolor{comment}{IS" AND ANY EXPRESS OR IMPLIED WARRANTIES, INCLUDING, BUT NOT LIMITED}
\textcolor{comment}{TO, THE IMPLIED WARRANTIES OF MERCHANTABILITY AND FITNESS FOR A}
\textcolor{comment}{PARTICULAR PURPOSE ARE DISCLAIMED. IN NO EVENT SHALL THE COPYRIGHT}
\textcolor{comment}{HOLDER OR CONTRIBUTORS BE LIABLE FOR ANY DIRECT, INDIRECT, INCIDENTAL,}
\textcolor{comment}{SPECIAL, EXEMPLARY, OR CONSEQUENTIAL DAMAGES (INCLUDING, BUT NOT}
\textcolor{comment}{LIMITED TO, PROCUREMENT OF SUBSTITUTE GOODS OR SERVICES; LOSS OF USE,}
\textcolor{comment}{DATA, OR PROFITS; OR BUSINESS INTERRUPTION) HOWEVER CAUSED AND ON ANY}
\textcolor{comment}{THEORY OF LIABILITY, WHETHER IN CONTRACT, STRICT LIABILITY, OR TORT}
\textcolor{comment}{(INCLUDING NEGLIGENCE OR OTHERWISE) ARISING IN ANY WAY OUT OF THE USE}
\textcolor{comment}{OF THIS SOFTWARE, EVEN IF ADVISED OF THE POSSIBILITY OF SUCH DAMAGE.}
\textcolor{comment}{******************************<GINKGO LICENSE>*******************************/}

\textcolor{preprocessor}{#include <iostream>}
\textcolor{preprocessor}{#include <map>}
\textcolor{preprocessor}{#include <string>}


\textcolor{preprocessor}{#include <omp.h>}
\textcolor{preprocessor}{#include <ginkgo/ginkgo.hpp>}


\textcolor{keyword}{extern} \textcolor{keywordtype}{void} stencil\_kernel(std::size\_t size, \textcolor{keyword}{const} \textcolor{keywordtype}{double} *coefs,
                           \textcolor{keyword}{const} \textcolor{keywordtype}{double} *b, \textcolor{keywordtype}{double} *x);


\textcolor{keyword}{class }StencilMatrix : \textcolor{keyword}{public} \hyperlink{classgko_1_1EnableLinOp}{gko::EnableLinOp}<StencilMatrix>,
                      \textcolor{keyword}{public} \hyperlink{classgko_1_1EnableCreateMethod}{gko::EnableCreateMethod}<StencilMatrix> \{
\textcolor{keyword}{public}:
    StencilMatrix(std::shared\_ptr<const gko::Executor> exec,
                  \hyperlink{namespacegko_a6e5c95df0ae4e47aab2f604a22d98ee7}{gko::size\_type} size = 0, \textcolor{keywordtype}{double} left = -1.0,
                  \textcolor{keywordtype}{double} center = 2.0, \textcolor{keywordtype}{double} right = -1.0)
        : \hyperlink{namespacegko}{gko}::EnableLinOp<StencilMatrix>(exec, \hyperlink{namespacegko}{gko}::dim<2>\{size\}),
          coefficients(exec, \{left, center, right\})
    \{\}

\textcolor{keyword}{protected}:
    \textcolor{keyword}{using} vec = \hyperlink{classgko_1_1matrix_1_1Dense}{gko::matrix::Dense<>};
    \textcolor{keyword}{using} coef\_type = \hyperlink{classgko_1_1Array}{gko::Array<double>};

    \textcolor{keywordtype}{void} apply\_impl(\textcolor{keyword}{const} \hyperlink{classgko_1_1LinOp}{gko::LinOp} *b, \hyperlink{classgko_1_1LinOp}{gko::LinOp} *x)\textcolor{keyword}{ const override}
\textcolor{keyword}{    }\{
        \textcolor{keyword}{auto} dense\_b = gko::as<vec>(b);
        \textcolor{keyword}{auto} dense\_x = gko::as<vec>(x);

        \textcolor{keyword}{struct }stencil\_operation : \hyperlink{classgko_1_1Operation}{gko::Operation} \{
            stencil\_operation(\textcolor{keyword}{const} coef\_type &coefficients, \textcolor{keyword}{const} vec *b,
                              vec *x)
                : coefficients\{coefficients\}, b\{b\}, x\{x\}
            \{\}

            \textcolor{keywordtype}{void} run(std::shared\_ptr<const gko::OmpExecutor>)\textcolor{keyword}{ const override}
\textcolor{keyword}{            }\{
                \textcolor{keyword}{auto} b\_values = b->get\_const\_values();
                \textcolor{keyword}{auto} x\_values = x->get\_values();
\textcolor{preprocessor}{#pragma omp parallel for}
                \textcolor{keywordflow}{for} (std::size\_t i = 0; i < x->\hyperlink{classgko_1_1LinOp_a31b3c003388eb0b95393154f68c2b98d}{get\_size}()[0]; ++i) \{
                    \textcolor{keyword}{auto} coefs = coefficients.get\_const\_data();
                    \textcolor{keyword}{auto} result = coefs[1] * b\_values[i];
                    \textcolor{keywordflow}{if} (i > 0) \{
                        result += coefs[0] * b\_values[i - 1];
                    \}
                    \textcolor{keywordflow}{if} (i < x->get\_size()[0] - 1) \{
                        result += coefs[2] * b\_values[i + 1];
                    \}
                    x\_values[i] = result;
                \}
            \}

            \textcolor{keywordtype}{void} run(std::shared\_ptr<const gko::CudaExecutor>)\textcolor{keyword}{ const override}
\textcolor{keyword}{            }\{
                stencil\_kernel(x->\hyperlink{classgko_1_1LinOp_a31b3c003388eb0b95393154f68c2b98d}{get\_size}()[0], coefficients.get\_const\_data(),
                               b->get\_const\_values(), x->get\_values());
            \}


            \textcolor{keyword}{const} coef\_type &coefficients;
            \textcolor{keyword}{const} vec *b;
            vec *x;
        \};
        this->get\_executor()->run(
            stencil\_operation(coefficients, dense\_b, dense\_x));
    \}

    \textcolor{keywordtype}{void} apply\_impl(\textcolor{keyword}{const} \hyperlink{classgko_1_1LinOp}{gko::LinOp} *alpha, \textcolor{keyword}{const} \hyperlink{classgko_1_1LinOp}{gko::LinOp} *b,
                    \textcolor{keyword}{const} \hyperlink{classgko_1_1LinOp}{gko::LinOp} *beta, \hyperlink{classgko_1_1LinOp}{gko::LinOp} *x)\textcolor{keyword}{ const override}
\textcolor{keyword}{    }\{
        \textcolor{keyword}{auto} dense\_b = gko::as<vec>(b);
        \textcolor{keyword}{auto} dense\_x = gko::as<vec>(x);
        \textcolor{keyword}{auto} tmp\_x = dense\_x->clone();
        this->apply\_impl(b, lend(tmp\_x));
        dense\_x->scale(beta);
        dense\_x->add\_scaled(alpha, lend(tmp\_x));
    \}

\textcolor{keyword}{private}:
    coef\_type coefficients;
\};


\textcolor{keywordtype}{void} generate\_stencil\_matrix(\hyperlink{classgko_1_1matrix_1_1Csr}{gko::matrix::Csr<>} *matrix)
\{
    \textcolor{keyword}{const} \textcolor{keyword}{auto} discretization\_points = matrix->\hyperlink{classgko_1_1LinOp_a31b3c003388eb0b95393154f68c2b98d}{get\_size}()[0];
    \textcolor{keyword}{auto} row\_ptrs = matrix->\hyperlink{classgko_1_1matrix_1_1Csr_a068e5158cf282fa977f0a137f8cd7f03}{get\_row\_ptrs}();
    \textcolor{keyword}{auto} col\_idxs = matrix->\hyperlink{classgko_1_1matrix_1_1Csr_a81c6294177a1be4873804c8a85a9fc64}{get\_col\_idxs}();
    \textcolor{keyword}{auto} values = matrix->\hyperlink{classgko_1_1matrix_1_1Csr_a929b0a194e6aeb1252b8e6781d162e83}{get\_values}();
    \textcolor{keywordtype}{int} pos = 0;
    \textcolor{keyword}{const} \textcolor{keywordtype}{double} coefs[] = \{-1, 2, -1\};
    row\_ptrs[0] = pos;
    \textcolor{keywordflow}{for} (\textcolor{keywordtype}{int} i = 0; i < discretization\_points; ++i) \{
        \textcolor{keywordflow}{for} (\textcolor{keyword}{auto} ofs : \{-1, 0, 1\}) \{
            \textcolor{keywordflow}{if} (0 <= i + ofs && i + ofs < discretization\_points) \{
                values[pos] = coefs[ofs + 1];
                col\_idxs[pos] = i + ofs;
                ++pos;
            \}
        \}
        row\_ptrs[i + 1] = pos;
    \}
\}


\textcolor{keyword}{template} <\textcolor{keyword}{typename} Closure>
\textcolor{keywordtype}{void} generate\_rhs(Closure f, \textcolor{keywordtype}{double} u0, \textcolor{keywordtype}{double} u1, \hyperlink{classgko_1_1matrix_1_1Dense}{gko::matrix::Dense<>} *rhs)
\{
    \textcolor{keyword}{const} \textcolor{keyword}{auto} discretization\_points = rhs->\hyperlink{classgko_1_1LinOp_a31b3c003388eb0b95393154f68c2b98d}{get\_size}()[0];
    \textcolor{keyword}{auto} values = rhs->\hyperlink{classgko_1_1matrix_1_1Dense_a3bc458e02fab8e4c9f60f70bd4d5a4f9}{get\_values}();
    \textcolor{keyword}{const} \textcolor{keyword}{auto} h = 1.0 / (discretization\_points + 1);
    \textcolor{keywordflow}{for} (\textcolor{keywordtype}{int} i = 0; i < discretization\_points; ++i) \{
        \textcolor{keyword}{const} \textcolor{keyword}{auto} xi = (i + 1) * h;
        values[i] = -f(xi) * h * h;
    \}
    values[0] += u0;
    values[discretization\_points - 1] += u1;
\}


\textcolor{keywordtype}{void} print\_solution(\textcolor{keywordtype}{double} u0, \textcolor{keywordtype}{double} u1, \textcolor{keyword}{const} \hyperlink{classgko_1_1matrix_1_1Dense}{gko::matrix::Dense<>} *u)
\{
    std::cout << u0 << \textcolor{charliteral}{'\(\backslash\)n'};
    \textcolor{keywordflow}{for} (\textcolor{keywordtype}{int} i = 0; i < u->\hyperlink{classgko_1_1LinOp_a31b3c003388eb0b95393154f68c2b98d}{get\_size}()[0]; ++i) \{
        std::cout << u->\hyperlink{classgko_1_1matrix_1_1Dense_ab83c739c1b11abaecc3bfd89506f6c9c}{get\_const\_values}()[i] << \textcolor{charliteral}{'\(\backslash\)n'};
    \}
    std::cout << u1 << std::endl;
\}


\textcolor{keyword}{template} <\textcolor{keyword}{typename} Closure>
\textcolor{keywordtype}{double} calculate\_error(\textcolor{keywordtype}{int} discretization\_points, \textcolor{keyword}{const} \hyperlink{classgko_1_1matrix_1_1Dense}{gko::matrix::Dense<>} *u,
                       Closure correct\_u)
\{
    \textcolor{keyword}{const} \textcolor{keyword}{auto} h = 1.0 / (discretization\_points + 1);
    \textcolor{keyword}{auto} error = 0.0;
    \textcolor{keywordflow}{for} (\textcolor{keywordtype}{int} i = 0; i < discretization\_points; ++i) \{
        \textcolor{keyword}{using} std::abs;
        \textcolor{keyword}{const} \textcolor{keyword}{auto} xi = (i + 1) * h;
        error +=
            \hyperlink{namespacegko_a57797fc0a00fd4b7ff34ca4bfc84bc51}{abs}(u->\hyperlink{classgko_1_1matrix_1_1Dense_ab83c739c1b11abaecc3bfd89506f6c9c}{get\_const\_values}()[i] - correct\_u(xi)) / 
      \hyperlink{namespacegko_a57797fc0a00fd4b7ff34ca4bfc84bc51}{abs}(correct\_u(xi));
    \}
    \textcolor{keywordflow}{return} error;
\}


\textcolor{keywordtype}{int} main(\textcolor{keywordtype}{int} argc, \textcolor{keywordtype}{char} *argv[])
\{
    \textcolor{keyword}{using} vec = \hyperlink{classgko_1_1matrix_1_1Dense}{gko::matrix::Dense<double>};
    \textcolor{keyword}{using} mtx = \hyperlink{classgko_1_1matrix_1_1Csr}{gko::matrix::Csr<double, int>};
    \textcolor{keyword}{using} cg = \hyperlink{classgko_1_1solver_1_1Cg}{gko::solver::Cg<double>};

    \textcolor{keywordflow}{if} (argc < 2) \{
        std::cerr << \textcolor{stringliteral}{"Usage: "} << argv[0] << \textcolor{stringliteral}{" DISCRETIZATION\_POINTS [executor]"}
                  << std::endl;
        std::exit(-1);
    \}

    \textcolor{keyword}{const} \textcolor{keywordtype}{unsigned} \textcolor{keywordtype}{int} discretization\_points =
        argc >= 2 ? std::atoi(argv[1]) : 100u;
    \textcolor{keyword}{const} \textcolor{keyword}{auto} executor\_string = argc >= 3 ? argv[2] : \textcolor{stringliteral}{"reference"};

    \textcolor{keyword}{const} \textcolor{keyword}{auto} omp = \hyperlink{classgko_1_1OmpExecutor_a33ca05fdd0fc928ee262fc9425304874}{gko::OmpExecutor::create}();
    std::map<std::string, std::shared\_ptr<gko::Executor>> exec\_map\{
        \{\textcolor{stringliteral}{"omp"}, omp\},
        \{\textcolor{stringliteral}{"cuda"}, \hyperlink{classgko_1_1CudaExecutor_a2718a92034350650ef406ffdb60db090}{gko::CudaExecutor::create}(0, omp)\},
        \{\textcolor{stringliteral}{"reference"}, gko::ReferenceExecutor::create()\}\};

    \textcolor{keyword}{const} \textcolor{keyword}{auto} exec = exec\_map.at(executor\_string);  \textcolor{comment}{// throws if not valid}
    \textcolor{keyword}{const} \textcolor{keyword}{auto} app\_exec = exec\_map[\textcolor{stringliteral}{"omp"}];

    \textcolor{keyword}{auto} correct\_u = [](\textcolor{keywordtype}{double} x) \{ \textcolor{keywordflow}{return} x * x * x; \};
    \textcolor{keyword}{auto} f = [](\textcolor{keywordtype}{double} x) \{ \textcolor{keywordflow}{return} 6 * x; \};
    \textcolor{keyword}{auto} u0 = correct\_u(0);
    \textcolor{keyword}{auto} u1 = correct\_u(1);

    \textcolor{keyword}{auto} rhs = vec::create(app\_exec, \hyperlink{structgko_1_1dim}{gko::dim<2>}(discretization\_points, 1));
    generate\_rhs(f, u0, u1, lend(rhs));
    \textcolor{keyword}{auto} u = vec::create(app\_exec, \hyperlink{structgko_1_1dim}{gko::dim<2>}(discretization\_points, 1));
    \textcolor{keywordflow}{for} (\textcolor{keywordtype}{int} i = 0; i < u->\hyperlink{classgko_1_1LinOp_a31b3c003388eb0b95393154f68c2b98d}{get\_size}()[0]; ++i) \{
        u->\hyperlink{classgko_1_1matrix_1_1Dense_a3bc458e02fab8e4c9f60f70bd4d5a4f9}{get\_values}()[i] = 0.0;
    \}

    cg::build()
        .with\_criteria(gko::stop::Iteration::build()
                           .with\_max\_iters(discretization\_points)
                           .on(exec),
                       \hyperlink{classgko_1_1stop_1_1ResidualNormReduction}{gko::stop::ResidualNormReduction<>::build}()
                           .with\_reduction\_factor(1e-6)
                           .on(exec))
        .on(exec)
        ->generate(
            StencilMatrix::create(exec, discretization\_points, -1, 2, -1))
        ->apply(lend(rhs), lend(u));

    print\_solution(u0, u1, lend(u));
    std::cout << \textcolor{stringliteral}{"The average relative error is "}
              << calculate\_error(discretization\_points, lend(u), correct\_u) /
                     discretization\_points
              << std::endl;
\}
\end{DoxyCodeInclude}
 